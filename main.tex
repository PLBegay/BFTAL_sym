\documentclass{report}
\usepackage[utf8]{inputenc}

\title{Bases formelles du TAL}
\author{Pierre-Léo Bégay}
%\date{October 2019}

\usepackage{natbib}
\usepackage{graphicx}
\usepackage{hyperref}
\usepackage{amsmath}
\usepackage{tikz}
\usepackage{subcaption}
\usepackage{tikz-qtree}
\usepackage{tikz-qtree-compat}
\usepackage{graphbox}
\usepackage{tabularx}

\usepackage[a4paper]{geometry}


\hypersetup{
    colorlinks=true,       % false: boxed links; true: colored links
    linkcolor=blue,          % color of internal links (change box color with linkbordercolor)
    citecolor=blue,        % color of links to bibliography
    filecolor=magenta,      % color of file links
    urlcolor=cyan           % color of external links
}
\usepackage{parskip}
\setlength{\parskip}{\baselineskip}%
\setlength{\parindent}{0pt}%
\usepackage{amsthm}
\usepackage{epigraph}
\setlength{\epigraphwidth}{0.6\textwidth}
%\theoremstyle{definition}
%\newtheorem{definition}{Définition}[section]
 
\theoremstyle{example}
\newtheorem{example}{Exemple}[section]
 
\theoremstyle{example}
\newtheorem{exercice}{Exercice}[section]
 
 
\theoremstyle{remark}
\newtheorem*{remark}{Remarque}

\usepackage{tcolorbox}
\tcbuselibrary{theorems}


\newtcbtheorem[number within=section]{definition}{Définition}%
{colback=blue!5,colframe=blue!30!black,fonttitle=\bfseries}{th}

\begin{document}

\maketitle

\epigraph{La théorie des automates est l'algèbre linéaire de l'informatique [...], connaissance de base, fondamentale, connue de tous et utilisée par tous, qui fait partie du paysage intellectuel depuis si longtemps qu'on ne l'y remarquerait plus}{Jacques Sakarovitch, dans l'avant-propos de Eléments de théorie des automates}
%\begin{abstract}
%    Initiation à la théorie des langages formels à destination des linguistes, se concentrant sur le troisième niveau de la hiérarchie de Chomsky. 
%\end{abstract}

\tableofcontents


\chapter{Notion de calculabilité}

%Pour une version plus détaillée de cette section, voir \cite{dowek}. 
Pour une version plus détaillée, voir \href{https://www.college-de-france.fr/site/xavier-leroy/inaugural-lecture-2018-11-15-18h00.htm}{la leçon inaugurale de Xavier Leroy au Collège de France}.

\section{Différents modèles de calcul}

L'informatique a vocation à automatiser les calculs afin de les faire réaliser par une machine plutôt que par des Humains. Avant d'automatiser les calculs, il s'agit donc de définir formellement de quoi on parle.

D'un point de vue programmation, 2 principaux modèles co-existent : les \textbf{machines de Turing} et le \textbf{$\lambda$-calcul}. Ces langages ne sont pas utilisables en pratique (\href{http://www.ens-lyon.fr/actualite/lecole/la-machine-de-turing-en-legos}{encore que}), mais posent les fondamentaux de ce qu'est un langage de programmation.

\paragraph{Machines de Turing} Les machines de Turing disposent d'une notion d'état, et d'une mémoire infinie modifiable. La notion d'état sera discutée en longueur dans la section \ref{automates}, mais correspond en gros à une mémoire spéciale, propre à chaque programme, qui peut être modifiée et consultée facilement, notamment pour s'orienter dans le \textit{flow} du programme\footnote{Pensez à une variable booleene "premierefois" que vous avez sans doute déjà utilisée dans un \texttt{if} pour accéder ou non à un cas particulier}. Ces traits rapprochement fortement les machines de Turing de la programmation impérative (C, langage machine, le coeur de Python etc). Les machines de Turing sont dues à \textbf{Alan Turing}.

\paragraph{$\lambda$-calcul} A la différence des machines de Turing, qui ont une approche quasiment mécanique (pour ne pas dire "bidouille") de l'exécution d'un programme, le $\lambda$-calcul est profondément mathématique. Tout n'y est que fonction, au point que ces dernières sont des objets comme les autres, notamment passables en arguments. On citera l'exemple classique d'une fonction qui reçoit une fonction de tri et une liste, et renvoie la liste triée selon la fonction fournie. Le $\lambda$-calcul est la base de la programmation fonctionnelle. Il a été crée par \textbf{Alonzo Church}.

\paragraph{Remarque} Les types en programmation impérative n'ont souvent qu'une valeur de garde-fou contre des opérations totalement absurdes, alors qu'ils ont une fonction beaucoup plus structurante (certaine.s diraient "contraignante") en programmation fonctionnelle. L'utilisation de fonctions comme arguments oblige par exemple à repenser les types et aller plus loin que les classiques \verb!bool!, \verb!int! et cie. On renverra encore une fois à la présentation de Xavier Leroy citée en introduction pour une meilleur vision d'ensemble.

Ces deux modèles ne forment pas l'alpha et l'omega de la calculabilité, qui contient de nombreux modèles plus ou moins exotiques, comme les fonctions $\mu$-récursives ou les automates cellulaires (voir à ce sujet \href{https://www.youtube.com/watch?v=S-W0NX97DB0}{la vidéo de la chaîne ScienceEtonnante}).

\paragraph{Thèse de Church (ou thèse de Church-Turing)} Church et Turing ont montré, dans les années 30, que les machines de Turing et le $\lambda$-calcul sont équivalents, dans le sens où toute fonction exprimable dans un modèle le sera dans l'autre. On dit que les modèles ont la même \textbf{expressivité}. Attention cependant, certaines fonctions très simples en $\lambda$-calcul seront un enfer à coder en machine de Turing, et inversement\footnote{Penser à la différence entre compétence et performance.}. Mais il reste remarquable que deux modèles fonctionnant de façons si orthogonales aient, au fond, la même puissance. 

Ce résultat est moins surprenant avec le recul, puisqu'on sait maintenant que tous la plupart des modèles de calcul non-triviaux sont équivalents, et forment la classe des modèles \textbf{Turing-complets}. Il est en effet possible de simuler, ou coder, les machines de Turing ou le $\lambda$-calcul dans les fonctions $\mu$-récursives ou les automates cellulaires, et inversement. Ce résultat s'étend à une armée de modèles, auxquels ils faut ajouter aujourd'hui des milliers de langages de programmation : Python, C, OCaml, Java, et même Makefile, Bash ou $\LaTeX$ sont bel et bien aussi expressifs les uns que les autres.

La thèse de Church peut même s'étendre à l'épistémologie ou à la philosophie, puisqu'elle peut sembler suggérer l'existence d'une notion "naturelle" et indépassable de calcul. On conseillera la lecture de \cite{dowek} pour une introduction à ces problématiques.


Le $\lambda$-calcul est utilisé en NLP, mais sera étudié dans un autre cours. Les machines de Turing quant à elles n'ont, en soi, pas d'intérêt pour la linguistique, mais on peut les affaiblir pour les rendre paradoxalement plus pertinentes.

\section{Décidabilité}

Au $XVII^{ème}$ siècle, Leibniz rêve d'une procédure permettant de déterminer automatiquement, \textit{via} un calcul, si une formule mathématique est vraie ou non. Leibniz se rendit compte que les bases formelles n'étaient alors pas disponibles, notamment la formalisation du calcul. Le problème réapparaît dans un cadre plus faorable, en 1928, lorsque Hilbert repose la question dans le cadre de son fameux programme de refondation des mathématiques. Le problème de la décision prend alors le doux nom d'\textbf{Entscheidungsproblem}. 

La réponse ne se fera pas (trop) attendre, et c'est un double non. Church et Turing publient en 1936, mais indépendament, des preuves qu'une telle procèdure, ou plutôt un tel programme, est impossible à écrire en machine de Turing et $\lambda$-calcul, et donc pour tout modèle de calcul équivalent. Le problème de la décision est \textbf{indécidable}, et il est loin d'être le seul.

\begin{theorem}{\textbf(Indécidabilité du problème de l'arrêt)}} Savoir si un programme termine est un problème indécidable.
\end{theorem}

\begin{proof}
On va procéder par l'absurde. La décidabilité du problème de l'arrêt signifierait qu'il existe un programme, appelé $A$, qui prend en argument un programme $P$ et un élément $x$, et renvoie \verb!true! si et seulement si $A(x)$ termine.

A partir de $A$, on peut constuire un autre programme appelé $B$, qui prend en argument un programme $P$, et termine si et seulement si $P(P)$ ne termine pas :\\
\verb!def B P := if (A P P) then (while true skip) else skip!.

Maintenant, appliquons $B$ à lui même. On utilisant la définition de B, on obtient\\ \verb!B(B) := if (A B B) then (while true skip) else skip!, ce qui veut dire que $B(B)$ termine si et seulement si $B(B)$ ne termine pas. On obtient donc un paradoxe, signifiant que notre seule hypothèse, l'existence de $A$, est fausse.
\end{proof}

\paragraph{Remarque} La preuve contient une bizarrerie, à savoir l'application d'un programme à lui-même ($B(B)$). Une telle chose est proscrite par l'utilisation de types, qui rendent la preuve donnée caduque. Il est cependant toujours possible de prouver l'indécidabilité de l'arrêt pour des programmes typés, de façon plus tordue cependant.

Au-delà de ces deux problèmes particulièrement connus, il existe une véritable armée de problèmes indécidables, comme le montre spécifié par le théorème de Rice :


\begin{theorem}{\textbf(Théorème de Rice)}} On appelle propriété sémantique non-triviale une propriété sur le comportement d'un programme telle qu'il existe au moins un exemple la respectant et un ne la respectant pas. Toute propriété sémantique non-triviale est indécidable.
\end{theorem}

\begin{proof}
On procède encore une fois par l'absurde, en supposant qu'il existe une propriété sémantique non-triviale $i$ décidable. Puisque $i$ est non-triviale, on sait qu'il existe $P_{i+}$ (resp. $P_{i-}$) un programme qui satisfait (resp. ne satisfait pas) la propriété $i$. On va montrer qu'il est alors possible de résoudre le problème de l'arrêt.

Soit un programme $P$ dont on veut vérifier qu'il termine sur l'argument $x$. On vérifie d'abord si $P$ satisfait la propriété $i$. Supposons, sans perte de généralité (il suffit sinon d'inverser les $+$ et $-$), que ce n'est pas le cas. On écrit alors un programme qui fait tourner $P(x)$, puis $P_{i+}$. On vérifie si le tout satisfait $i$. Si c'est le cas, on sait que $P(x)$ a fini. A l'inverse, si ce n'est pas le cas, $P_{i+}$ n'a pas été atteint, ce qui veut dire que $P(x)$ n'a pas fini.

L'existence supposée de la décidabilité propriété sémantique non-triviale permet de 
résoudre le problème de l'arrêt, pourtant indécidable. Il n'existe donc pas de telle propriété.\end{proof}

Intuitivement, tous ces problèmes sont indécidables sur tout modèle Turing-complet car ces derniers sont trop puissants, ou expressifs\footnote{Comme le disent tous les oncles Ben du monde, un grand pouvoir (expressif) implique une grande indécidabilité}. Commence alors un jeu consistant à affaiblir les modèles pour qu'on puisse décider des propriétés à leur sujet, sans pour autant qu'ils en deviennent trivial. Ce cours va s'intéresser à la famille des automates, particulièrement adaptée à l'analyse linguistique.%De cette recherche découle notamment un modèle qui 1) est plus faible 2) concerne les langages, et donc les linguistes : les automates finis. 

Puisqu'on veut les analyser, on va d'abord définir la notion de langages - et donc de mots dans la section \ref{langages}. On étudiera dans la section \ref{regex} un outil pour les décrire et manipuler, les expressions régulières. On pourra ensuite s'intéresser aux automates les plus simples dans la section \ref{automates}, ainsi qu'à la notion connexe de grammaires formelles en \ref{grammaires}. Enfin, on étudiera en \ref{hierarchie} la façon dont ces différents outils s'imbriquent, et on \textit{teasera} la suite du cours, en évoquant des extensions des automates.


\chapter{Langages}
\label{langages}

On définit d'abord la notion de mot, nécessaire à celle de langage. On verra ensuite comment décrire des langages à l'aide de notations ensemblistes, révisant ces dernières par la même occasion.

\section{Mots}

\begin{definition}{Mot}{}Un \textbf{mot} est une suite de lettres tirées d'un alphabet donné. L'ensemble des mots sur un alphabet $\Sigma$ est noté $\Sigma^*$.
\end{definition}

\begin{example}
Etant donné l'alphabet $\Sigma = \{a,b,c\}$, on peut construire une infinité de mots parmi lesquels 

\begin{itemize}
    \item $abc$
    \item $aab$
    \item $cc$
    \item $abcabcacbacbacbacbabcabcabcabcabcabcabc$
    \item $a$
\end{itemize}

\end{example}


\paragraph{Remarque} On va s'intéresser ici à des langages et mots complètement abstraits, en général composés uniquement de $a$, $b$ et $c$.

\begin{definition}{Mot vide}{}
Une suite de lettres peut être de longueur zéro, formant alors \textbf{le mot vide}. Quel que soit l'alphabet, ce dernier sera noté $\epsilon$.
\end{definition}

\begin{definition}{Concaténation}{}
\label{concat}
L'opération de \textbf{concaténation}, notée $.$, consiste tout simplement à "coller" deux mots.
\end{definition}

\begin{example}{Quelques concaténations :}

\begin{itemize}
    \item $ab.c = abc$
    \item $ab.ba = abba$
\end{itemize}
\end{example}

 De plus, pour tout mot $w$, 

\[
    w.\epsilon = \epsilon.w = w
\]

\paragraph{Remarque} Les algébristes enthousiastes remarqueront que $(\Sigma^*,.,\epsilon)$ forme un monoïde libre de base $\Sigma$


\begin{definition}{Longueur d'un mot}{}
Etant donné un mot $w$, on note sa \textbf{longueur} $|w|$.
\end{definition}

\begin{example}
Tout naturellement,
\begin{itemize}
    \item $|abc| = 3$
    \item $|abba| = 4$
    \item $|c| = 1$
    \item $|\epsilon| = 0$
\end{itemize}
\end{example}

\begin{definition}{Principe d'induction sur un mot}{}
Etant donnée une propriété $P$ sur les mots. Si on a\\

\begin{enumerate}
\item $P(\epsilon)$ (cad. que $P$ est vraie pour le mot vide)
\item $\forall w, \forall c \in \Sigma, (P(w) \rightarrow P(c.w))$ (cad. que si $P$ est vraie pour un mot, alors elle reste vraie si on rajoute n'importe quelle lettre à gauche du mot)\\
\end{enumerate}

Alors la propriété $P$ est vraie pour tout mot $w$.
\end{definition}

\paragraph*{Remarque} Est également valide le principe d'induction où, dans le cas récursif, la lettre est rajoutée à droite du mot plutôt qu'à sa gauche.

On va s'entraîner à utiliser ce principe d'induction en prouvant deux lemmes qui ne le nécessitaient sans doute pas :

\begin{lemma}
$\forall w \in \Sigma^*, |w| \geq 0$, cad. que tout mot a une longueur positive.
\end{lemma}
\begin{proof}
On procède par induction sur $w$.

Dans le cas de base, $w = \epsilon$. On a donc $|w| = |\epsilon| = 0 \geq 0$.

Dans le cas récursif, $w = c.w'$ avec $c \in \Sigma$ et on suppose $|w'| \geq 0$. On a $|c.w'| = 1 + |w'| \geq |w'| \geq 0$.
\end{proof}

On va s'entraîner à utiliser ce principe d'induction en prouvant deux lemmes qui n'en nécessitaient sans doute pas tant :

\begin{lemma}
$\forall w \in \Sigma^*, |w| \geq 0$, cad. que tout mot a une longueur positive.
\end{lemma}
\begin{proof}
On procède par induction sur $w$.

Dans le cas de base, $w = \epsilon$. On a donc $|w| = |\epsilon| = 0 \geq 0$.

Dans le cas récursif, $w = c.w'$ avec $c \in \Sigma$ et on suppose $|w'| \geq 0$. On a $|c.w'| = 1 + |w'| \geq |w'| \geq 0$.
\end{proof}

\begin{lemma}
Etant donnés deux mots $w_1$ et $w_2$, $|w_1.w_2| = |w_1| + |w_2|$. 
\end{lemma}

\begin{proof}
On procède par induction sur $w_1$.

Dans le cas de base, $w_1 = \epsilon$. On a donc $|w_1.w_2| = |\epsilon.w_2| = |w_2| = 0 + |w_2| = |w_1| + |w_2|$.

Dans le cas récursif, $w_1 = c.w_1'$ avec $c \in \Sigma$ et on suppose $|w_1'.w_2| = |w_1'| + |w_2|$. On a
\begin{tabular}{cll}
&&\\
& $|w_1.w_2|$ & \\
$=$ & $|c.w_1'.w_2|$ & par définition de $w_1$ \\
$=$&  $1 + |w_1'.w_2|$ & par définition de $|.|$ \\
$=$& $1 + (|w_1'| + |w_2|)$ & par hypothèse d'induction \\
$=$& $(1 + |w_1'|) + |w_2|$ & par associativité de l'addition \\
$=$& $|c.w_1'| + |w_2|$ & par définition de $|.|$ \\
$=$& $|w_1| + |w_2|$ & par définition de $w_1$
\end{tabular}

On a donc bien nos deux conditions pour le raisonnement par induction.
\end{proof}


\begin{definition}{Nombre d'occurrences d'une lettre}{}
Etant donné un mot $w$ et une lettre $a$, on note $\mathbf{|w|_a}$ le nombre de $a$ dans $w$.
\end{definition}

\begin{example}
On a 
\begin{itemize}
    \item $|abc|_a = 1$
    \item $|abba|_b = 2$
    \item $|c|_a = 0$
    \item $|\epsilon|_a = 0$
\end{itemize}
\end{example}

\begin{definition}{Préfixe}{}
Un mot $p$ est un \textbf{préfixe} du mot $w$ ssi $\exists v, w = p.v$, cad. ssi $w$ \textit{commence} par $p$.
\end{definition}

\begin{definition}{Suffixe}{}
Un mot $s$ est un \textbf{suffixe} du mot $w$ ssi $\exists v, w = v.s$, cad. ssi $w$ \textit{finit} par $p$.
\end{definition}

\begin{example}
Le mot $abba$ admet comme préfixes $\epsilon$, $a$, $ab$, $abb$ et $abba$. Ses suffixes sont, quant à eux, $\epsilon$, $a$, $ba$, $bba$ et $abba$.
\end{example}

\begin{lemma}
$\forall w \in \Sigma^*, \epsilon$ et $w$ sont des préfixes de $w$
\end{lemma}
 
\begin{proof}
Pour $\epsilon$, il suffit de prendre $v = w$. A l'inverse, en prenant $v = \epsilon$, on voit que $w$ est son propre préfixe.
\end{proof}

\begin{lemma}
$\forall w \in \Sigma^*, \epsilon$ et $w$ sont des suffixes de $w$
\end{lemma}

\begin{proof}
Analogue au lemme précédent.
\end{proof}


\begin{exercice}\label{expref}
Combien de préfixes et suffixes admet un mot $w$ quelconque ?
\end{exercice}

\begin{definition}{Facteur}{}
Un mot $f$ est un \textbf{facteur} du mot $w$ ssi $\exists v_1 v_2, w = v_1.f.v_2$, cad. ssi $f$ apparaît dans $w$.
\end{definition}

\begin{example}
Les facteurs du mot $abba$ sont $\epsilon$, $a$, $b$, $ab$, $ba$, $abb$, $bba$ et $abba$. 
\end{example}

\begin{lemma}
$\forall w \in \Sigma^*$, $\epsilon$ et $w$ sont des facteurs de $w$.
\end{lemma}

\begin{proof}
Pour $\epsilon$, il suffit de prendre $v_1 = w$ et $v_2 = \epsilon$ (ou l'inverse) et la condition est trivialement vérifiée. Pour $w$, on prend $v_1 = v_2 = \epsilon$.
\end{proof}

\begin{exercice}
Donner l'ensemble des facteurs du mot $abbba$.
\end{exercice}

\begin{exercice} \label{exfact}(*)
Donner la borne la plus basse possible du nombre de facteurs d'un mot $w$. Donner un mot d'au moins 3 lettres dont le nombre de facteurs est exactement la borne donnée.
\end{exercice}

\begin{definition}{Sous-mot}{}
Un mot $s$ est un \textbf{sous-mot} du mot $w$ ssi $w = v_0s_0v_1s_1v_2...s_nv_n$ et $s = s_0s_1...s_n$, cad. ssi $w$ est "$s$ avec (potentiellement) des lettres en plus".
\end{definition}

\begin{example}\label{ex5} On \underline{souligne} les lettres originellement présentes dans le sous-mot :
\begin{itemize}
   \item $ab$ est un sous-mot de $b\underline{a}a\underline{b}$, qu'on pourrait aussi voir comme $ba\underline{ab}$
   \item $abba$ est un sous-mot de $ba\underline{a}abaa\underline{bb}a\underline{a}$.
   \item $ba$ \underline{n}'est \underline{pas} un sous-mot de $aaabbb$ (l'ordre du sous-mot doit être préservé dans le mot) 
\end{itemize}
\end{example}


\begin{lemma}
$\forall w \in \Sigma^*$, $\epsilon$ et $w$ sont des sous-mots de $w$.
\end{lemma}

\begin{proof}
Pour $\epsilon$, il suffit de prendre $n = 0$, $s_0 = \epsilon$, $v_0 = w$ et $v_1 = \epsilon$ (ou l'inverse) et la condition est trivialement vérifiée. Pour $w$, on prend $n = 0$, $s_0 = w$ et $v_0 = v_1 = \epsilon$.
\end{proof}

\begin{exercice}
Montrer que tout facteur d'un mot en est également un sous-mot. A l'inverse, montrer qu'un sous-mot n'est pas forcément un facteur. 
\end{exercice}

\begin{exercice}
Donner toutes les façons de voir $abba$ comme sous-mot de $baaabaabbaa$ (cf. exemple \ref{ex5}).
\end{exercice}

\begin{exercice}
Donner l'ensemble des sous-mots de $abba$
\end{exercice}

\begin{exercice}\label{exssmot} (*)
Donner la borne la plus basse possible du nombre de sous-mots d'un mot $w$. Donner un mot dont le nombre de sous-mots est exactement la borne donnée.
\end{exercice}

\begin{exercice} (*) Dans l'exercice \ref{expref}, on demande le nombre exact de préfixes et suffixes d'un mot, alors que dans les exercices \ref{exfact} et \ref{exssmot}, on demande une borne, pourquoi ?
\end{exercice}

\section{Langage}

\begin{definition}{Langage}{}
Un langage, c'est un ensemble de mots. 
\end{definition}

On distingue donc d'entrée les deux langages extrêmes : $\mathbf{\Sigma^*}$, l'ensemble (infini) de tous les mots formés à partir de $\Sigma$, et $\mathbf{\emptyset}$, le langage / ensemble vide, qui se caractèrise comme ne contenant aucun élément.

\paragraph{Remarque} Ne surtout pas confondre $\emptyset$ et $\{\epsilon\}$. Le premier est un ensemble vide, contenant donc \textbf{0} élément, trandis que le second contient \textbf{1} élément, le mot ide.

\begin{definition}{Produit de langages}{}
Le produit de deux langages $L_1$ et $L_2$, noté $L_1 . L_2$, renvoie l'ensemble des mots composés d'un mot de $L_1$ puis d'un de $L_2$ :\\
$L_1 . L_2 = \{w_1 . w_2 ~|~ w_1 \in L_1 ~\wedge~w_2 \in L_2\}$

Il s'agit d'un cas particulier de produit d'ensembles (cf. définition \ref{ensprod}.
\end{definition}

\begin{example}
Soit $L_1 = \{ab,b,\epsilon\}$ et $L_2 = \{a,b,aa\}$, on a\\
\begin{tabular}{cl} 
    & $L_1 . L_2$ \\
 $=$& $\{ab.a,ab.b,ab.aa,b.a,b.b,b.aa,\epsilon.a,\epsilon.b,\epsilon.aa\}$ \\
 $=$& $\{aba,abb,abaa,ba,bb,baa,a,b,aa\}$
\end{tabular}
\end{example}

Le produit de langage peut être itéré\footnote{Concrètement, les puissances sur les langages ont le même sens que sur les nombres, avec la multiplication remplacée par la concaténation} :

\begin{eqnarray*}
L^0 = \{\epsilon\}\\
L^{n+1} = L^n . L
\end{eqnarray*}

Les langages disposent en plus d'un opérateur spécial :


\begin{definition}{Etoile de Kleene}{}
Soit $L$ un langage. On note $L^*$ la concaténation de n'importe quel nombre de mots apparaissant dans $L$, cad.

\[
L^* = \bigcup_{n \in \mathbf{N}} L^n = L^0 \cup L^1 \cup L^2 \cup ... 
\]
\end{definition}

\begin{example}
Soit $L = \{aa,b\}$, on a \\
\begin{tabular}{rl}
$L^* =$& $\{epsilon\} $ \\
 $\cup $& $\{aa,b\}$ \\
$\cup$ & $ \{aaaa,aab,baa,bb\}$ \\
 $\cup $ & $ \{aaaaaa,aaaab,aabaa,aabb,baaaa,baab,bbaa,bbb\}$ \\
 $\cup$ & ...
\end{tabular}
\end{example}

La question maintenant est maintenant de savoir comment on définit et parle de langages précis et plus "intermédiaires" que les deux précédents. En tout généralité, les ensembles peuvent être définis de façon \textbf{extensionnelle} ou \textbf{intentionnelle}. 

\begin{definition}{Définition extensionnelle d'un ensemble}{}
On \textbf{définit extensionnellement un ensemble} en en donnant la liste des éléments. L'ensemble vide se note quant à lui $\emptyset$.
\end{definition}

\begin{example} On définit par exemple l'ensemble (sans intérêt) suivant :
\[
    A = \{b, aca, abba\}
\]
\end{example}

Les définitions extensionnelles ont le mérite d'être pour le moins simples, mais pas super pratiques quand il s'agit de définir des ensembles avec un nombre infini d'éléments, comme l'ensemble des mots de longueur pair. 

\begin{definition}{Définition intensionnelle d'un ensemble}{}
On \textbf{définit intensionnellement un ensemble} à l'aide d'une propriété que tous ses éléments satisfont. Étant donnés une propriété $Q(x)$ (typiquement représentée sous la forme d'une formule logique) et un ensemble $A$, on note $\{x \in A~|~Q(x)\}$ l'ensemble des éléments de $A$ qui satisfont $P$. Si l'ensemble $A$ est évident dans le contexte, on s'abstiendera de le préciser.
\end{definition}

\begin{example}
On peut définir l'ensemble des mots de longueur paire $\{w \in \Sigma^*~|~|w|~pair\}$.
\end{example}

Si les définitions intentionnelles permettent, contrairement aux extensionnelles, de dénoter des ensembles contenant une infinité de mots, elles sont avant tout un outil théorique. En effet, une propriété comme "$|w|$ paire" ne dit rien à un ordinateur en soi, et doit donc être définie formellement. Se pose alors la question d'un langage pour les propriétés.

Plusieurs logiques équipées des bonnes primitives peuvent être utilisées, mais les traductions sont rarement très agréables. Certaines propriétés nécessitent en effet de ruser contre le langage, voire sont impossibles à formaliser dans certaines logiques. Il existe heureusement un outil qui va nous aider, avec le premier problème du moins.

\chapter{Expressions régulières}
\label{regex}
Les expressions régulières permettent définir de façon finie - et relativement intuitive - "la forme" des mots d'un langage, potentiellement infini. On en présentera d'abord le lexique et l'idée générale à l'aide d'exemples, puis on en définira formellement la syntaxe et la sémantique.


\chapter{Automates}
\label{automates}

\chapter{Grammaires formelles}
\label{grammaires}

\chapter{Théorème de Kleene et hiérarchie de Chomsky}
\label{hierarchie}

\bibliographystyle{plain}
\bibliography{references}

\appendix


\chapter{Rappels mathématiques}
%\section{Lexique }

%\paragraph{}On rappelle quelques notions et notations de base.

\section{Logique}

\subsection{Raisonnement par l'absurde}
\label{abs}

\begin{definition}{Raisonnement par l'absurde}{}
Un \textbf{raisonnement par l'absurde} consiste à prouver une chose en 1) supposant son contraire et 2) montrer que ça fout tout en l'air. Plus formellement, pour prouver $P$, on suppose $\neg P$ et on montre que ça nous permet de déduire $\bot$, ce qui veut dire soit que la logique est incohérente, soit que $\neg P$ est fausse, et donc que $P$ est vraie.
\end{definition}

\begin{example}
Imaginons une situation où les rues sont sèches, et où on voudrait prouver qu'il n'a pas plu. On suppose alors l'inverse, c'est-à-dire qu'il a plu. Or, s'il a plu, les routes sont mouillées. On obtient alors que 1) les routes sont mouillées et 2) les routes ne sont pas mouillées, ce qui est un paradoxe. La seule hypothèse faite étant le fait qu'il a plu, elle doit être fausse.
\end{example}

\begin{example}
On veut prouver qu'il existe une infinité de nombres premiers. On suppose l'inverse, cad. qu'il y en a un ensemble fini $\{p_1,...,p_n\}$. Soit $n = 1 + \prod_{i \in [1 - n]} p_i = 1 + p_1 \times ... \times p_n$. $n$, comme tout nombre, admet au moins un diviseur premier. 

Or, $n$ est strictement plus grand que tout nombre premier et ne peut donc pas en être un. De plus, pour tout $i \in [1 - n]$, $\frac{n}{p_i} = p_1 \times ... \times p_{i-1} \times p_{i+1} \times ... \times p_n + \frac{1}{p_i}$. Tout nombre premier étant $\geq 2$, $\frac{1}{p_i}$ ne forme pas un entier, et donc $\frac{n}{p_i}$ non plus.

On obtient une contradiction, notre hypothèse sur la finitude des nombres premiers est donc fausse.
\end{example}



\section{Ensembles}

\subsection{Ensemble des parties}

Soit un ensemble $X$, on note $P(X)$ (parfois $2^X$) l'ensemble de ses parties, cad l'ensemble des ensembles formés à partir d'éléments de $X$.

\begin{example}
Soit $X = \{x,y,z\}$, alors - en classant les éléments par leur cardinal -

\begin{tabular}{llll}
 $P(X) = \{$ & $\emptyset$, & & \\
& $\{x\}$,&$\{y\}$,&$\{z\}$,\\
&$\{x,y\}$,& $\{x,z\}$,&$\{y,z\}$\\
& $\{x,y,z\}$ & &$~~~~~~~~\}$.
\end{tabular}
\end{example}

\begin{lemma}
$\forall X, \emptyset \in P(X) \wedge X \in P(X)$.
\end{lemma}

\begin{lemma}
$\forall X, |P(X)| = 2^{|X|}$.
\end{lemma}

\begin{proof}
Pour générer l'ensemble des sous-ensembles de $X$, on choisit de prendre ou non chaque élément de $X$. On a donc $\underbrace{2 \times 2 \times ... \times 2}_{\textrm{n fois}}$ choix, d'où les $2^{|X|}$ au total.
\end{proof}

\subsection{Opérations entre ensembles}

\begin{definition}{Produit d'ensembles}{}
\label{ensprod}
Soit deux ensembles $E_1$ et $E_2$, contenant respectivement des éléments de types $\tau_1$ et $\tau_2$. Soit également $\cdot$ une opération de type $\tau_1 \rightarrow \tau_2 \rightarrow \tau_3$, cad. une opération qui prend en argument gauche un élément de type $\tau_1$ et à droite un argument de type $\tau_2$ et renvoie un objet de type $\tau_3$, alors 

\[
E_1 \cdot E_2 = \{x \cdot y~|~x \in E_1~\wedge~y \in E_2\}
\]
\end{definition}

Dit autrement, un produit d'ensembles renvoie l'ensemble des combinaisons d'éléments de deux ensembles \textit{via} une opération fournie. Si l'opération $\cdot$ est un endormorphisme, cad. qu'elle est de type $\tau \rightarrow \tau \rightarrow \tau$, alors on peut itérer le produit de la façon suivante :

\begin{eqnarray*}
E^0 = \{1\} ~~~~~~~~~~~~~~~~ \textrm{Où $1$ est l'élément neutre de $\tau$} \\
E^{n+1} = E^n \cdot E
\end{eqnarray*}

Cette notion très générale ne doit pas être confondue avec

\begin{definition}{Produit cartésien}{}
Soit deux ensembles $E_1$ et $E_2$, ne contenant pas forcément des éléments de même type, alors

\[
E_1 \times E_2 = \{(x,y)~|~x \in E_1~\wedge~y \in E_2\}
\]
\end{definition}

Le produit cartésien, noté $\times$, renvoie l'ensemble des couples d'éléments de deux ensembles donnés. Il s'agit d'un cas particulier du produit d'ensemble, où l'opération est la "mise en couple". Cette opération ne pouvant pas être un endormorphisme, le produit cartésien ne peut être itéré.




\end{document}