\chapter{Expressions régulières}
\label{regex}

\section{Lexique et idée générale}

\subsection{Les lettres et $\epsilon$, la base}

\subsection{$.$, la concaténation}

\subsection{*, l'itération}

\begin{exercice}
Donner 5 autres mots appartenant au langage dénotée par l'expression de l'exemple \ref{ex1}. 
\end{exercice}

\begin{exercice}
Pourquoi le changement de formulation dans les exemples \ref{ex0} et \ref{ex1} par rapport aux exemples précédents ("c'est à dire $\{x, y, z ... \}$" qui devient "contenant notamment $x$, $y$ ou $z$") ?
\end{exercice}

\subsection{$+$, la disjonction}

\begin{exercice}
Donner un mot acceptant deux dérivations avec la regex de l'exemple \ref{ex3} (justifier en donnant les dérivations). Existe-t-il un autre mot admettant plusieurs dérivations ?
\end{exercice}

\begin{exercice}
Existe-t-il un mot acceptant plusieurs dérivations pour la regex de l'exemple \ref{ex4} ?
\end{exercice}

\begin{exercice}
Donner un mot accepté par la regex de l'exemple \ref{ex4} mais pas celle de l'exemple \ref{ex3}. Est-il possible de trouver un mot qui, à l'inverse, est accepté par la deuxième mais pas la première ?
\end{exercice}


\begin{exercice} (*)
Exprimer, en langue naturelle et de façon concise, le langage dénoté par la regex de l'exemple \ref{ex2}. Traduire ensuite ce langage en une regex non-ambiguë, c'est-à-dire où il n'y aura qu'une dérivation pour chaque mot.
\end{exercice}

\section{Syntaxe}


\section{Sémantique}
\label{resem}

\subsection{Les cas de base}

\subsection{Sémantique de la concaténation}

\subsection{Sémantique de la disjonction}

\subsection{Sémantique de l'itération}


\section{Mise en application}

\subsection{Quelques astuces}

\begin{exercice}
Donner une \textit{regex} pour les mots qui commencent par $a$.
\end{exercice}

\begin{exercice}
Donner une \textit{regex} pour les mots qui finissent par $b$.
\end{exercice}

\begin{exercice}
Donner une \textit{regex} pour les mots qui commencent par $a$ finissent par $b$.
\end{exercice}


\begin{exercice}
Donner une \textit{regex} pour les mots de longueur paire.
\end{exercice}

\begin{exercice}
Donner une \textit{regex} pour les mots de longueur impaire qui contiennent au moins 4 lettres.
\end{exercice}

\begin{exercice}
Donner une \textit{regex} pour les mots de longueur impaire, qui contiennent au moins 4 lettres, comment par $a$ et finissent par $b$.
\end{exercice}


\subsection{Syntaxe en pratique}


