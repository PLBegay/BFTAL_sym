\documentclass{article}[11pt]


%\documentclass[runningheads,a4paper]{llncs}
%\usepackage{hyperref}
%\usepackage{ amssymb }


%\gasset{frame=false} % switch to true to add frames
%\parindent=0pt

%\usepackage{bussproofs}
%\usepackage{varwidth}
%\usepackage{xspace}
%\usepackage{verbatim}

\usepackage{float} 
\usepackage[lmargin=4cm, rmargin=4cm, tmargin=3cm, bmargin=3cm]{geometry}
\usepackage{algpseudocode}
\usepackage{pdfpages}

\usepackage{url}
\usepackage{multirow} 
\usepackage{mathtools}
\usepackage{stmaryrd}
\DeclarePairedDelimiter{\ceil}{\lceil}{\rceil}
\DeclarePairedDelimiter{\floor}{\lfloor}{\rfloor} 
%\usepackage{xspace}
\usepackage{latexsym,amsmath,amsfonts,amssymb,stmaryrd}
\let\proof\relax
\let\endproof\relax 
\usepackage{amsthm} 
\usepackage{tikz}
\usepackage{tikz-qtree}
\usepackage{tikz-qtree-compat}

\usepackage{listings}
\usepackage{chngpage}
\DeclareMathAlphabet{\mathpzc}{OT1}{pzc}{m}{it}

\usepackage{ amssymb }
\usepackage{soul}

\usepackage{tipa}
\usepackage{stmaryrd}

\usepackage{verbatim} 
\usepackage{epsfig} 
\usepackage{graphics}
\usepackage{ mathrsfs }
\usepackage{hyperref}
\usepackage{multicol}

\usepackage{epigraph}


% \usepackage{tipa}
\usepackage{graphicx}   
%\usepackage{url}
\usepackage{wrapfig}  
\usepackage{bm}   
\usepackage{epstopdf}  
\usepackage{ upgreek }
\usepackage[all,cmtip]{xy}

\usepackage{natbib}
% \usepackage{float} 
% \usepackage[lofdepth,lotdepth]{subfig}
%\usepackage{graphicx}
\usepackage[T1]{fontenc} 
\usepackage[utf8]{inputenc}
\usepackage{etoolbox}
\usepackage{textcomp}

\usepackage{mdwtab}
\usepackage{syntax} 

\renewcommand{\syntleft}{}          % do not display '<' associated with variable, for example <A>
\renewcommand{\syntright}{}         % do not display '>' associated with variable, for example <A>


\makeatletter 
\patchcmd{\maketitle}{\@copyrightspace}{}{}{}
\makeatother 
\usepackage{mathpartir}
%\usepackage{enumitem}
%\usepackage{supertabular} 
%\usepackage{soul}
\usepackage[all]{xy}
\usepackage{xifthen}
\usepackage{placeins} 
\usepackage{amsthm}
\usepackage{amsmath}


\usepackage{pgf}
\usepackage{tikz}
\usetikzlibrary{arrows,automata}
\tikzset{initial text={}}
\usetikzlibrary{calc,shapes.multipart,chains,arrows}
%prevents second paragraph indentations 
%\usepackage{parskip}
% \usepackage{floatrow}
\usepackage{tabularx} % in the preamble
\usepackage{bm}
\usepackage{caption}
\usepackage{subcaption} 


\newtheorem{theorem}{Theorem}
\newtheorem{lemma}[theorem]{Lemma}


\setlength{\bibsep}{0pt plus 0.5ex}





\newenvironment{mathprooftree}
  {\varwidth{.9\textwidth}\centering\leavevmode}
  {\DisplayProof\endvarwidth}


\newcommand{\qm}{\overline{Q}\overline{m}}

\newcommand{\myparagraph}[1]{\paragraph{#1}\mbox{}\\}
\newcommand\myeq{\stackrel{\mathclap{\normalfont\mbox{\scriptsize{def}}}}{=}}

\title{DM de Bases formelles du TAL}

\author{Pierre-Léo Bégay}
\date{À me rendre le 1er mai 2020}
\theoremstyle{definition}
\newtheorem{exmp}{Exemple}
   

\begin{document}
 
\maketitle
\pagestyle{empty} %
\thispagestyle{empty}

%% Attention: pas plus d'un recto-verso!
% Ne conservez pas les questions

\section{Automates}

\subsection{Complétion}

\paragraph{Question} Donnez un automate qui reconnaît le complémentaire du langage reconnu par celui-ci : 


\begin{figure}[H]
\centering

\begin{tikzpicture}[->,>=stealth',shorten >=1pt,auto,node distance=2.5cm,
                    semithick]
  \tikzstyle{every state}=[fill=white,text=black]
  \tikzstyle{place}=[rectangle,draw=black,fill=white, minimum size=7mm]


  \node[initial, state, accepting] (0)                    {$0$};
  \node[state] (1)      [right of=0]                {$1$};
  \node[state,accepting] (2)      [below of=0]                {$2$};
  \node[state] (3)      [right of=2]                {$3$};
 
 
  \path %(I) edge[loop above]              node {$a,b$} (I)

(0) edge      []        node {$a$} (1)

(1) edge      []        node {$a$} (3)

(3) edge      []        node {$a$} (0)
(2) edge      []        node {$a,b$} (3)

(0) edge      []        node {$b$} (2)
;
\end{tikzpicture}
\end{figure}


\subsection{Déterminisation}

\paragraph{Question 1} Déterminisez l'automate suivant :



\begin{figure}[H]
\centering

\begin{tikzpicture}[->,>=stealth',shorten >=1pt,auto,node distance=2.5cm,
                    semithick]
  \tikzstyle{every state}=[fill=white,text=black]
  \tikzstyle{place}=[rectangle,draw=black,fill=white, minimum size=7mm]


  \node[initial, state] (0)                    {$0$};
  \node[state] (1)      [above right of=0]                {$1$};
  \node[state] (2)      [right of=1]                {$2$};
  \node[state] (3)      [right of=2]                {$3$};
  \node[state,accepting] (4)      [right of=3]                {$4$};
  \node[state] (5)      [below right of=0]                {$5$};
  \node[state] (6)      [right of=5]                {$6$};
  \node[state] (7)      [right of=6]                {$7$};
 \node[state,accepting] (8)      [right of=7]   
 {$8$};
 \node[state,initial] (9)      [above left of=7]   
 {$9$};
 \node[state,accepting] (10)      [above right of=7]   
 {$10$};
  
 
  \path %(I) edge[loop above]              node {$a,b$} (I)
  (2) edge      [loop above]        node {$a,b$} (2)
(8) edge      [loop below]        node {$a,b$} (8)

(0) edge      []        node {$a$} (1)

(9) edge      []        node {$b$} (7)

(7) edge      []        node {$b$} (10)

(1) edge      []        node {$a$} (2)
(2) edge      []        node {$a$} (3)

(3) edge      []        node {$a$} (4)
(0) edge      []        node {$a$} (5)

(5) edge      []        node {$a,b$} (6)
(6) edge      []        node {$a$} (7)

(7) edge      []        node {$a$} (8);
\end{tikzpicture}
\end{figure}

\paragraph{Question 2} Donnez une expression rationnelle décrivant le langage reconnu par l'automate.

\subsection{Minimisation}

\paragraph{Question 1} Minimisez l'automate suivant :


\begin{figure}[H]
\centering

\begin{tikzpicture}[->,>=stealth',shorten >=1pt,auto,node distance=2.5cm,
                    semithick]
  \tikzstyle{every state}=[fill=white,text=black]
  \tikzstyle{place}=[rectangle,draw=black,fill=white, minimum size=7mm]


  \node[initial, state,accepting] (0)                    {$0$};
  \node[state] (1)      [right of=0]                {$1$};
  \node[state,accepting] (2)      [right of=1]                {$2$};
  \node[state] (3)      [right of=2]                {$3$};
  \node[state,accepting] (4)      [below of=3]                {$4$};
  \node[state] (5)      [left of=4]                {$5$};
  \node[state,accepting] (6)      [left of=5]                {$6$};
  \node[state] (7)      [left of=6]                {$7$};
  \path %(I) edge[loop above]              node {$a,b$} (I)
(0) edge      [bend left=20]        node {$a$} (3)
(0) edge      [bend right,out=240,in=-120,looseness=1.5]        node {$b$} (4)

(1) edge      []        node {$a$} (2)
(1) edge      []        node {} (7)

(2) edge      []        node {$a$} (3)
(2) edge      []        node {} (6)

(3) edge      []        node {$a$} (4)
(3) edge      []        node {} (5)

(4) edge      [bend left=20]        node {$a$} (7)
(4) edge      [bend right,out=240,in=-120,looseness=1.5]        node {$b$} (0)

(5) edge      []        node {$a$} (6)
(5) edge      []        node {$b$} (3)

(6) edge      []        node {$a$} (7)
(6) edge      []        node {$b$} (2)

(7) edge      []        node {$a$} (0)
(7) edge      []        node {$b$} (1);
\end{tikzpicture}
\end{figure}


\paragraph{Question 2} Quel est le langage reconnu par l'automate ?

\paragraph{Bonus} Essayez d'expliquer comment l'automate non-minimal reconnaissait le langage, et en quoi il n'était pas optimal

\paragraph{} Dans l'automate original, on s'amuse donc à sauter de façon très gratuite entre états équivalents sans grand sens. Par exemple, quand on lit un $b$ en 1 on va en 7 et inversement au lieu de rester en place, parce que pourquoi pas. Il fonctionne donc comme quatre copies de sa version minimale mélangées de façon chaotique.

\subsection{Automate produit}

\paragraph{Question} Donnez un automate qui reconnaît l'intersection des langages reconnus par les deux automates suivant :


\begin{figure}[H]
\centering

\begin{tikzpicture}[->,>=stealth',shorten >=1pt,auto,node distance=2.5cm,
                    semithick]
  \tikzstyle{every state}=[fill=white,text=black]
  \tikzstyle{place}=[rectangle,draw=black,fill=white, minimum size=7mm]


  \node[initial, state,accepting] (0)                    {$0$};
  \node[state] (1)      [right of=0]                {$1$};
  \node[state,accepting] (2)      [below of=0]                {$2$};
  \node[state] (3)      [right of=2]                {$3$};
  \node[initial,state] (0bis)      [right of=1]                {$0$};
  \node[state,accepting] (1bis)      [right of=0bis]                {$1$};
  \node[state] (2bis)      [right of=1bis]                {$2$};
  
  \path %(I) edge[loop above]              node {$a,b$} (I)
(0) edge      []        node {$a$} (1)
(1) edge      []        node {$a,b$} (2)
(2) edge      []        node {$b$} (3)
(2) edge      [bend left]        node {$a$} (0)
(3) edge      [loop right]        node {$a,b$} (3)
(0) edge      [bend left]        node {$b$} (2)

(0bis) edge      [loop above]        node {$a$} (0bis)
(2bis) edge      [bend left]        node {$a,b$} (1bis)
(1bis) edge      [loop above]        node {$b$} (1bis)
(0bis) edge      []        node {$b$} (1bis)
(1bis) edge      [bend left]        node {$a$} (2bis)

;
\end{tikzpicture}
\end{figure}


\section{Langages intrinsèquement ambigus}

Dans cet exercice, on s'intéressera \textbf{uniquement} aux \textbf{grammaires de type 2} (ou grammaires algébriques). On rappelle que ces grammaires n'acceptent que les règles de la forme $A \rightarrow \gamma$, avec $\gamma \in (\Sigma \cup V)^*$. \newline \newline
\noindent
Vos réponses devront être accompagnées d'une justification légère (de l'ordre d'une ou deux phrases) expliquant comment la grammaire donnée génère le langage de la question.

\paragraph*{Indice} Il n'est pas interdit de penser au cours sur les propriétés de clôture des langages réguliers.

\subsection*{Question 0}

Donnez une grammaire qui reconnaît le langage $L_0 = \{a^nb^n~|~n \in \mathbb{N}\}$

\subsection*{Question 1}

Donnez une grammaire qui reconnaît le langage $L_1 = \{a^nba^n~|~n \in \mathbb{N}\}$


\subsection*{Question 2}

Donnez une grammaire qui reconnaît le langage $L_2 = \{a^nba^m~|~n,m \in \mathbb{N}$ et $ n \geq m\}$

\subsection*{Question 3}

Donnez une grammaire qui reconnaît le langage 

\paragraph*{}$L_3 = \{a^nba^mba^pba^q~|~n,m,p,q \in \mathbb{N}, n \geq m$ et $p \geq q\}$

\subsection*{Question 4}

Donnez une grammaire qui reconnaît le langage 

\paragraph*{}$L_4 = \{a^nba^mba^pba^q~|~n,m,p,q \in \mathbb{N}, n \geq q$ et $m \geq p\}$

\subsection*{Question 5}

Donnez une grammaire qui reconnaît le langage 

\paragraph*{}$L_5 = \{a^nba^mba^pba^q~|~n,m,p,q \in \mathbb{N}$ et $((n \geq m$ et $p \geq q)$ ou $(n \geq q$ et $m \geq p))\}$

\subsection*{Question 6}

\paragraph*{}Donnez, dans la grammaire de la question 5, deux dérivations différentes d'un même mot de $L_5$ (pas de justification demandée)

\newpage
\section{Grammaire mystère}

\noindent
Soit la grammaire de type 0\footnote{Le langage décrit est lui-même de type 1, ce qui veut dire qu'on pourrait le faire avec une grammaire contextuelle, mais c'est plus \textit{simple} en s'accordant le luxe d'une type 0} suivante : $\big \langle \{a,b,\#\}, \{S,S',A,B,\$\},S,\{$

\begin{enumerate}

\item $S \rightarrow \$_GS'\$_D$
\item $S' \rightarrow aAS'$
\item $S' \rightarrow bBS'$
\item $S' \rightarrow \epsilon$
\item $Aa \rightarrow aA$
\item $Ab \rightarrow bA$
\item $Ba \rightarrow aB$
\item $Bb \rightarrow bB$
\item $\$_Ga \rightarrow a\$_G$
\item $\$_Gb \rightarrow b\$_G$
\item $A\$_D \rightarrow \$_Da$
\item $B\$_D \rightarrow \$_Db$
\item $\$_G\$_D \rightarrow \#\}\ \big \rangle$

\end{enumerate}

\noindent
Expliquez, en des termes très simples et \textit{naturels}, le langage décrit par cette grammaire. Justifiez votre réponse en expliquant succinctement le fonctionnement de la grammaire (au moins les grandes étapes).

\paragraph*{Indice} Plutôt que d'essayer de deviner le langage engendré en fixant longuement les règles, dérivez\footnote{Notez que pour les grammaire de type 0 (et 1), la traduction des dérivations en arbre n'est plus possible. Vous devrez donc faire des dérivations `plates`, comme on faisait au début du chapitre sur les grammaires} quelques mots au hasard et voyez s'ils n'ont pas l'air de partager une propriété intéressante. C'est beaucoup plus facile de vérifier qu'une grammaire a une propriété donnée que de l'inférer. 




\end{document} 
  


 