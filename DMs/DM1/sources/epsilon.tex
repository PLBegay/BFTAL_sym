\documentclass{article}[11pt]


%\documentclass[runningheads,a4paper]{llncs}
%\usepackage{hyperref}
%\usepackage{ amssymb }


%\gasset{frame=false} % switch to true to add frames
%\parindent=0pt

%\usepackage{bussproofs}
%\usepackage{varwidth}
%\usepackage{xspace}
%\usepackage{verbatim}


\usepackage[lmargin=4cm, rmargin=4cm, tmargin=3cm, bmargin=3cm]{geometry}
\usepackage{algpseudocode}


\usepackage{url}
\usepackage{multirow} 
\usepackage{mathtools}
\usepackage{stmaryrd}
\DeclarePairedDelimiter{\ceil}{\lceil}{\rceil}
\DeclarePairedDelimiter{\floor}{\lfloor}{\rfloor} 
%\usepackage{xspace}
\usepackage{latexsym,amsmath,amsfonts,amssymb,stmaryrd}
\let\proof\relax
\let\endproof\relax 
\usepackage{amsthm} 
\usepackage{tikz}
\usepackage{tikz-qtree}
\usepackage{tikz-qtree-compat}

\usepackage{listings}
\usepackage{chngpage}
\DeclareMathAlphabet{\mathpzc}{OT1}{pzc}{m}{it}

\usepackage{ amssymb }
\usepackage{soul}

\usepackage{tipa}
\usepackage{stmaryrd}

\usepackage{verbatim} 
\usepackage{epsfig} 
\usepackage{graphics}
\usepackage{ mathrsfs }
\usepackage{hyperref}
\usepackage{multicol}

\usepackage{epigraph}


% \usepackage{tipa}
\usepackage{graphicx}   
%\usepackage{url}
\usepackage{wrapfig}  
\usepackage{bm}   
\usepackage{epstopdf}  
\usepackage{ upgreek }
\usepackage[all,cmtip]{xy}

\usepackage{natbib}
% \usepackage{float} 
% \usepackage[lofdepth,lotdepth]{subfig}
%\usepackage{graphicx}
\usepackage[T1]{fontenc} 
\usepackage[utf8]{inputenc}
\usepackage{etoolbox}
\usepackage{textcomp}

\usepackage{mdwtab}
\usepackage{syntax} 

\renewcommand{\syntleft}{}          % do not display '<' associated with variable, for example <A>
\renewcommand{\syntright}{}         % do not display '>' associated with variable, for example <A>


\makeatletter 
\patchcmd{\maketitle}{\@copyrightspace}{}{}{}
\makeatother 
\usepackage{mathpartir}
%\usepackage{enumitem}
\usepackage{code}
%\usepackage{supertabular} 
%\usepackage{soul}
\usepackage[all]{xy}
\usepackage{xifthen}
\usepackage{placeins} 
\usepackage{amsthm}
\usepackage{amsmath}


\usepackage{pgf}
\usepackage{tikz}
\usetikzlibrary{arrows,automata}
\tikzset{initial text={}}
\usetikzlibrary{calc,shapes.multipart,chains,arrows}
%prevents second paragraph indentations 
%\usepackage{parskip}
% \usepackage{floatrow}
\usepackage{tabularx} % in the preamble
\usepackage{bm}
\usepackage{caption}
\usepackage{subcaption} 

%\input{mac}
\newcommand{\im}[1]{\ensuremath{#1}}

\newcommand{\costs}[1]{$\vert #1 \vert \leq N$}

\newcommand{\kw}[1]{\im{\mathtt{#1}}}
\renewcommand{\ar}{\im{\rightarrow}}
\newcommand{\ca}{\im{\curvearrowright}}
\newcommand{\la}{\im{\leftarrow}}
\newcommand{\ra}{\im{\rightarrow}}
%\newcommand{\ts}{\vdash}
\newcommand{\gac}{\eqslantless}
\newcommand{\eac}{\diamond}
\newcommand{\lac}{\eqslantgtr}
\newcommand{\bottom}{\perp}

%% Frequently used letters lett
\newcommand{\mcx}{\im{\mathcal{X}}}
\newcommand{\mce}{\im{\mathcal{E}}}
\newcommand{\mcv}{\im{\mathcal{V}}}
\newcommand{\mcl}{\im{\mathcal{L}}}
\newcommand{\mct}{\im{\mathcal{T}}}
\newcommand{\mbbl}{\im{\mathbb{L}}}
\newcommand{\mbbp}{\im{\mathbb{P}}}

\newcommand{\closure}[1]{\im{\overline{#1}}}
\newcommand{\pmeet}[2]{\im{{#1} \cap {#2}}}

\renewcommand{\hole}{\im{\Box}}

\newcommand{\powerset}[1]{\im{2^{{#1}}}}
\newcommand{\ancst}[3]{\im{{#1} \stackrel{#3}{\leadsto} {#2}}}

\newcommand{\set}[1]{\im{\{{#1}\}}}

\newcommand{\mmax}{\ensuremath{\mathsf{max}}}

%%%%%%%%%%%%%%%%%%%%%%%%%%%%%%%%%%%%%%%%%%%%%%%%%%%%%%%%
% Comments
\newcommand{\omitthis}[1]{}

% Misc.
\newcommand{\etal}{\textit{et al.}}
\newcommand{\bump}{\hspace{3.5pt}}

% Text fonts
\newcommand{\tbf}[1]{\textbf{#1}}
%\newcommand{\trm}[1]{\textrm{#1}}

% Math fonts
\newcommand{\mbb}[1]{\mathbb{#1}}
\newcommand{\mbf}[1]{\mathbf{#1}}
\renewcommand{\mit}[1]{\mathit{#1}}
\newcommand{\mrm}[1]{\mathrm{#1}}
\newcommand{\mtt}[1]{\mathtt{#1}}
\newcommand{\mcal}[1]{\mathcal{#1}}
\newcommand{\mfrak}[1]{\mathfrak{#1}}
\newcommand{\msf}[1]{\mathsf{#1}}
\newcommand{\mscr}[1]{\mathscr{#1}}

% Text mode
\newenvironment{nop}{}{}

% Math mode
\newenvironment{sdisplaymath}{
\begin{nop}\small\begin{displaymath}}{
\end{displaymath}\end{nop}\ignorespacesafterend}
\newenvironment{fdisplaymath}{
\begin{nop}\footnotesize\begin{displaymath}}{
\end{displaymath}\end{nop}\ignorespacesafterend}
\newenvironment{smathpar}{
\begin{nop}\small\begin{mathpar}}{
\end{mathpar}\end{nop}\ignorespacesafterend}
\newenvironment{fmathpar}{
\begin{nop}\footnotesize\begin{mathpar}}{
\end{mathpar}\end{nop}\ignorespacesafterend}
\newenvironment{alignS}{
\begin{nop}\begin{align}}{
\end{align}\end{nop}\ignorespacesafterend}
\newenvironment{salignS}{
\begin{nop}\small\begin{align}}{
\end{align}\end{nop}\ignorespacesafterend}
\newenvironment{falignS}{
\begin{nop}\footnotesize\begin{align*}}{
\end{align}\end{nop}\ignorespacesafterend}

% Stack formatting
\newenvironment{stackAux}[2]{%
\setlength{\arraycolsep}{0pt}
\begin{array}[#1]{#2}}{
\end{array}}
\newenvironment{stackCC}{
\begin{stackAux}{c}{c}}{\end{stackAux}}
\newenvironment{stackCL}{
\begin{stackAux}{c}{l}}{\end{stackAux}}
\newenvironment{stackTL}{
\begin{stackAux}{t}{l}}{\end{stackAux}}
\newenvironment{stackTR}{
\begin{stackAux}{t}{r}}{\end{stackAux}}
\newenvironment{stackBC}{
\begin{stackAux}{b}{c}}{\end{stackAux}}
\newenvironment{stackBL}{
\begin{stackAux}{b}{l}}{\end{stackAux}}

 
%% \makeatletter
%% \newcommand\definitionname{Lemma}
%% \newcommand\listdefinitionname{Proofs of Lemmas and Theorems}
%% \newcommand\listofdefinitions{%
%%   \section*{\listdefinitionname}\@starttoc{def}}
%% \makeatother



\newtheoremstyle{athm}{\topsep}{\topsep}%
      {\upshape}%         Body font
      {}%         Indent amount (empty = no indent, \parindent = para indent)
      {\bfseries}% Thm head font
      {}%        Punctuation after thm head
      {.8em}%     Space after thm head (\newline = linebreak)
      {\thmname{#1}\thmnumber{ #2}\thmnote{~\,(#3)}
% \addcontentsline{Lemma}{Lemma}
%   {\protect\numberline{\thechapter.\thelemma}#1}
      % \ifstrempty{#3}%
      {\addcontentsline{def}{section}{#1~#2~#3}}%
      % {\addcontentsline{def}{subsection}{\theathm~#3}}
\newline}%         Thm head spec

 \theoremstyle{athm}


% \newtheoremstyle{break}
%   {\topsep}{\topsep}%
%   {\itshape}{}%
%   {\bfseries}{}%
%   {\newline}{}%
% \theoremstyle{break}

%There are some problems with llncs documentcalss, so commenting these out until i find a solution
\newtheorem{Theorem}{Theorem}

%\spnewtheorem{thm1}[theorem]{Theorem}{\bfseries}{\upshape}
%\newenvironment{Theorem}[1][]{\begin{thm1}\iffirstargument[#1]\fi\quad\\}{\end{thm1}}

 \newtheorem{Lemma}[Theorem]{Lemma}
 \newtheorem{Conjecture}{Conjecture}
 \newtheorem{Corollary}[Theorem]{Corollary}
 \newtheorem{Definition}{Definition}
 \newtheorem{Proposition}[Theorem]{Proposition}
 \newtheorem{Assumption}[Theorem]{Assumption}

% BNF symbols 
\newcommand{\bnfalt}{{\bf \,\,\mid\,\,}}
\newcommand{\bnfdef}{{\bf ::=~}}

%% Highlighting
\newcommand{\hlm}[1]{\mbox{\hl{$#1$}}}

%% Provenance modes
\newcommand{\modificationProvenance}{{\bf MP}}
\newcommand{\updateProvenance}{{\bf UP}}

%Lemmas
\newcommand{\lemref}[1]{Lemma \ref{#1}} %name and number

\usepackage{enumitem}
\setenumerate{listparindent=\parindent}

\newlist{enumih}{enumerate}{3}
\setlist[enumih]{label=\alph*),before=\raggedright, topsep=1ex, parsep=0pt,  itemsep=1pt }

\newlist{enumconc}{enumerate}{3}
\setlist[enumconc]{leftmargin=0.5cm, label= \arabic*.  , topsep=1ex, parsep=0pt,  itemsep=3pt }

\newlist{enumsub}{enumerate}{3}
\setlist[enumsub]{ leftmargin=0.7cm, label= \textbf{subcase} \bf \arabic*: }

\newlist{mainitem}{itemize}{3}
\setlist[mainitem]{ leftmargin=0.7cm , label= {\bf Case} }
 
% substitution
\newcommand{\tightoverset}[2]{%
  \mathop{#2}\limits^{\vbox to -.5ex{\kern-0.75ex\hbox{$#1$}\vss}}}

\newcommand{\subst}[3]{{#3}[{#1}/{#2}]} 
\newcommand{\substall}[3]{{#1}[{#2_i}/{#3_i}]_{_{i=1}}^{^n}}
\newcommand{\substfrom}[5]{{#1}[\tightoverset{\longrightarrow}{{#2_i}/{#3_i}}]_{_{i=#4}}^{^#5}}

%Replay
\newcommand{\replay}{\curvearrowright}
\newcommand{\extract}[1]{\kw{extract}(#1)}
\newcommand{\lreplay}{\overset{l}{\curvearrowright}}
\newcommand{\mapreplay}{\overset{map}{\curvearrowright}}

% Helpful shortcuts
\newcommand{\ie}{i.e.~}
\newcommand{\eg}{e.g.~}

% NEW POPL15 PAPER IGNORE ABOVE FOR NOW

%Types
\newcommand{\tmubig}[2]{\big(#1\big)^{#2}}
\newcommand{\tmu}[2]{(#1)^{#2}}
\newcommand{\tzero}[1]{(#1)^{\vzero}}
\newcommand{\tone}[1]{(#1)^{\vone}}
\newcommand{\tint}{\kw{int}}
\newcommand{\tbool}{\kw{bool}}
\newcommand{\tarr}[3]{#1 \xrightarrow{#3} #2}
\newcommand{\tarrs}[3]{\s{#1} \xrightarrow{\s{#3}} \s{#2}}
\newcommand{\tprod}[2]{#1 \times #2}
\newcommand{\tlist}[3]{\kw{list}\left[#1\right]^{#2}\,#3}
\newcommand{\tlists}[4]{\kw{list}\left[#4{#1}\right]^{#4{#2}}\,#4{#3}}
\newcommand{\tlistL}[3]{\kw{list}\left[#1\right]^{#2}#3}
\newcommand{\tlistb}[1]{\kw{list}\;#1}
\newcommand{\tunit}{\kw{unit}}
\newcommand{\tcint}[1]{\kw{N}[#1]}
\newcommand{\tcarr}[2]{#1 \mathrel{\rightarrow} #2}
\newcommand{\tcprod}[2]{#1 \mathrel{\wedge} #2}
\newcommand{\tforallK}[3]{\forall#1\overset{#3}{::}K.~ #2}
\newcommand{\mytforall}[4]{\forall#1\overset{#3}{::}#4.~ #2}
\newcommand{\tforallG}[4]{\forall#1\overset{#3}{{::}}{#2}.~ #4}
\newcommand{\tforall}[3]{\forall#1\overset{#3}.~ #2}
%\newcommand{\texists}[2]{\exists#1{::} K.~ #2}
\newcommand{\texists}[2]{\exists#1.~ #2}
\newcommand{\hastype}[2]{#1 : #2}
\newcommand{\subtype}[2]{#1\sqsubseteq#2}
\newcommand{\eqtype}[2]{#1\equiv#2}

%Kinds
% \newcommand{\ksize}{\iota}
% \newcommand{\kvar}{\upsilon}
% \newcommand{\kcost}{\omega}
\newcommand{\ksize}{\mathbb{N}}
\newcommand{\kvar}{\mathbb{V}}
\newcommand{\kcost}{\mathbb{R}}
\newcommand{\kfun}[2]{#1\mbox{\ra} #2}
\newcommand{\kfunmon}[2]{#1\xrightarrow{\mbox{mon}} #2}
\newcommand{\kk}{K}
\newcommand{\hk}[2]{#1 \mathrel{::} #2}
\newcommand{\kinded}[1]{#1 \mathrel{::} K}\newcommand{\sized}[1]{#1 \mathrel{::} \iota}
\newcommand{\Int}{\text{Int}}
\newcommand{\Bool}{\text{Bool}}
%Variations 

\newcommand{\scond}[3]{\im{#1}\mathrel{\mbox{?}}\im{#2}\mathrel{\colon}\im{#3}}
%% \newcommand{\vzero}{{0_{\mu}}}
%% \newcommand{\vone}{{1_{\mu}}}
\newcommand{\vzero}{\small{\mathbb{S}}}
\newcommand{\vone}{\small{\mathbb{C}}}
\newcommand{\slam}[2]{\underline{\lambda}#1.#2}
\newcommand{\sapp}[2]{{#1}({#2})}
\newcommand{\spower}[2]{#1^{#2}}
\newcommand{\ssum}[4]{\sum\limits_{#1=#2}^{#3}#4}
\newcommand{\smin}[2]{\kw{min}({#1},{#2})}
\newcommand{\smax}[2]{\kw{max}(#1,#2)}
%Sizes
\newcommand{\szero}{0}
\newcommand{\sone}{1}
\newcommand{\splus}[2]{#1 + #2}
\newcommand{\ssucc}[1]{#1 {+} \sone}
\newcommand{\sminus}[2]{#1 - #2}
\newcommand{\sdiv}[2]{\frac{#1}{#2}}
\newcommand{\smult}[2]{#1\cdot#2}
\newcommand{\splusone}[1]{#1+1}
\newcommand{\sceil}[1]{\ceil*{#1}}
\newcommand{\sfloor}[1]{\floor*{#1}}
\newcommand{\size}[1]{|#1|}
\newcommand{\slog}[1]{\kw{log}_2(#1)}
% \newcommand{\snlog}[1]{\kw{ln}}



%Constraints
\newcommand{\sat}[1]{\models#1}
\newcommand{\ceq}[2]{#1\mathrel{\doteq}#2}
\newcommand{\cleq}[2]{#1 \mathop{\leq} #2}
\newcommand{\clt}[2]{#1 \mathop{<} #2}
\newcommand{\cgt}[2]{#1 \mathop{>} #2}
\newcommand{\ceqz}[1]{#1 \mathrel{\doteq} 0}
\newcommand{\cneg}[1]{\mathop{\neg}#1}
\newcommand{\cand}[2]{#1 \wedge #2}

\newcommand{\blank}[2][100]{\hfil\penalty#1\hfilneg }

%TERMS
\newcommand{\const}[1]{\kw{#1}}
\newcommand{\ic}{\kw{n}}
\newcommand{\bc}{\kw{b}}
\newcommand{\true}{\kw{true}}
\newcommand{\false}{\kw{false}}
\newcommand{\ei}{\im{\mathtt{i}}}
\newcommand{\pair}[2]{\im{({#1},{#2})}}
\newcommand{\fst}[1]{\im{\kw{fst}~{#1}}}
\newcommand{\snd}[1]{\im{\kw{snd}~{#1}}}
\newcommand{\nil}{\im{\kw{nil}}}
%\newcommand{\cons}[2]{\im{\kw{cons}({#1}, {#2})}}
\newcommand{\cons}[2]{\im{\kw{cons}(#1,\,#2)}}

\newcommand{\fix}[3]{\im{\kw{fix}~~#1(#2).\, #3}}
\newcommand{\zero}{\kw{0}}
\renewcommand{\succ}[1]{\kw{succ}~#1}
\newcommand{\Lam}[1]{\Lambda.\, #1}
\newcommand{\lam}[2]{\lambda #1.\,#2}
\newcommand{\App}[1]{#1[]}
\newcommand{\AppM}[1]{#1[*]}
%\newcommand{\pack}[1]{\kw{pack}~#1}
\newcommand{\pack}[1]{\kw{pack}~#1} % ~\kw{as}~ #2

%\newcommand{\unpack}[3]{\kw{unpack}~#1~\kw{as}~#2~\kw{in}~#3}
\newcommand{\unpack}[3]{\kw{unpack}~#1~\kw{as}~#2~\kw{in}~#3}
\newcommand{\unpackin}[2]{\kw{unpack}~#1~\kw{as}~#2~\kw{in}}
%\newcommand{\casel}[5]{\kw{case_L}~#1~\kw{of}~\kw{nil}~\ra~#2~|~\kw{cons}(#3,#4)~\ra~#5}
\newcommand{\casel}[5]{\kw{case_L}~#1~\kw{of}~\kw{nil}~\ra~#2~|~\kw{cons}(#3,#4)~\ra~#5}

\newcommand{\caseof}[1]{\kw{case_L}~#1~\kw{of}}
\newcommand{\ofnil}[1]{~~\kw{nil}~\ra#1}
\newcommand{\ofcons}[3]{|~\kw{cons}(#1,~#2)~\ra~#3}
%aligned case
\newcommand{\caselAl}[7]{\kw{case_L}~#1~\kw{of}~ \kw{nil}~\ra~#2~
\newline|~\kw{cons}(#3,[#5][#6]~#4)~\ra~#7}
\newcommand{\casen}[4]{\kw{case_N}~#1~\kw{of}~\kw{0}~\ra~#2~\vert~\kw{succ}(#3)~\ra~#4}
\newcommand{\letin}[3]{\im{\kw{let}~{#1}\:=\:{#2}~ \kw{in}~{#3}}}
\newcommand{\letx}[2]{\im{\kw{let}~{#1}\:=\:{#2}~ \kw{in}}}
\newcommand{\app}[2]{\im{{#1}~{#2}}}
\newcommand{\plus}[2]{#1 + #2}
\newcommand{\mult}[2]{#1 * #2}
\newcommand{\equal}[2]{#1 = #2}
\newcommand{\zip}[2]{\kw{zip}({#1,#2)}}
\newcommand{\intToBool}[1]{\kw{intToBool}({#1)}}
\newcommand{\unit}{()}
\newcommand{\ccircle}{\circ}
\newcommand{\prim}{\zeta}
\newcommand{\primApp}[2]{#1~#2}
%PATTERNS
\newcommand{\pkeep}[1]{\kw{keep(#1)}}
\newcommand{\pnil}{\kw{nil}}
\newcommand{\punit}{()}
\newcommand{\pnull}{\kw{null}}
\newcommand{\pstable}[1]{\kw{stable}(#1)}
\newcommand{\pzero}{\zero}
\newcommand{\pcint}[1]{\im{{#1}}}
\newcommand{\psucc}[1]{\im{\kw{succ}~{#1}}}
\newcommand{\prepl}[2]{\kw{replace(#1,#2)}}
\newcommand{\ppair}[2]{\im{{(#1},{#2)}}}
\newcommand{\pfst}[1]{\im{\kw{fst}~{#1}}}
\newcommand{\psnd}[1]{\im{\kw{snd}~{#1}}}
\newcommand{\pcons}[2]{\im{\kw{cons}({#1},#2)}}
\newcommand{\pLam}[1]{\Lambda.#1}
\newcommand{\plet}[3]{\im{\kw{let}~{#1}\:=\:{#2}~\kw{in}~{#3}}}
\newcommand{\pfix}[3]{\kw{fix}~#1(#2).#3}
\newcommand{\ppack}[1]{\kw{pack}~#1} % ~\kw{as}~ #2
\newcommand{\pApp}[1]{#1[]}
\newcommand{\pprimApp}[2]{#1~#2}

\newcommand{\papp}[2]{#1 ~#2}
\newcommand{\pcasel}[5]{\kw{case_L}~#1~\kw{of}~\kw{nil}~\ra~#2~|~\kw{cons}(#3,#4)~\ra~#5}
\newcommand{\pcasen}[4]{\kw{case_N}~#1~\kw{of}~\kw{0}~\ra~#2~\vert~\kw{succ}~#3~\ra~#4}
\newcommand{\peq}[2]{#1\equiv#2}
\newcommand{\peqn}[2]{#1\equiv\pnull}
\newcommand{\punpack}[3]{\kw{unpack}~#1~\kw{as}~#2~\kw{in}~#3}

\newcommand{\Mytest}[1]{%
\begin{tabular}{c}
    #1 \\\hline
\end{tabular}%
} 
%% Trace expression pairs used for replay semantics
\newcommand{\trexp}[2] { \bm{\langle} #1, #2 \bm{\rangle}}
\newcommand{\run}[3]{#1 \red #2, #3}
\newcommand{\repexp}[2] { \bm{\langle} #1, #2 \bm{\rangle}}
\newcommand{\trmapexp}[4] { \bm{\langle} #1, #2,#3, #4 \bm{\rangle}}
\newcommand{\trfilterexp}[4] { \bm{\langle} #1, #2,#3, #4 \bm{\rangle}}

\newcommand{\mktrace}[1]{\im{{#1}}}
\newcommand{\tic}{\kw{n}}
\newcommand{\tbc}{\kw{b}}
\newcommand{\trbool}{\mktrace{b}}
\newcommand{\trv}{\mktrace{\cdot}}
\newcommand{\tr}[1]{\kw{#1}}
\newcommand{\trunit}{\mktrace{\unit}}
\newcommand{\trint}{\mktrace{i}}
\newcommand{\trzero}{\mktrace{\zero}}
\newcommand{\trtrue}{\mktrace{true}}
\newcommand{\trfalse}{\mktrace{false}}
\newcommand{\trcons}[2]{\mktrace{\kw{cons}({#1},{#2})}}
\newcommand{\trpair}[2]{\mktrace{({#1},{#2})}}
\newcommand{\treq}[2]{\mktrace{{#1} = {#2}}}
\newcommand{\trfst}[1]{\mktrace{{\kw{fst}}~#1}}
\newcommand{\trsucc}[1]{\mktrace{{\kw{succ}}~#1}}
\newcommand{\trsnd}[1]{\mktrace{{\kw{snd}}~#1}}
\newcommand{\trlet}[3]{\im{\kw{let}~{#1}\:=\:{#2}~\kw{in}~{#3}}}
\newcommand{\trfix}[3]{\mktrace{\kw{fix}#1(#2).#3}}
\newcommand{\trapp}[3]{\mktrace{\kw{app}(#1,#2,#3)}}
\newcommand{\trnil}{\mktrace{\kw{nil}}}
\newcommand{\trcasez}[2]{\mktrace{\kw{case_{\szero}}(#1,#2)}}
\newcommand{\trcases}[2]{\mktrace{\kw{case_s}(#1,#2)}}
\newcommand{\trcasen}[2]{\mktrace{\kw{case_{nil}}(#1,#2)}}
\newcommand{\trcasec}[2]{\mktrace{\kw{case_{cons}}(#1,#2)}}
\newcommand{\trLam}[1]{\mktrace{\Lambda.#1}}
\newcommand{\trApp}[2]{\kw{iapp}(#1,#2)}
\newcommand{\trpack}[1]{\kw{pack}~#1}
\newcommand{\trunpack}[3]{\kw{unpack}~#1~\kw{as}~#2~\kw{in}~#3}
\newcommand{\trprim}[3]{\kw{primApp}(#1,#2,#3)}

\newcommand{\iprim}[1]{\widehat{#1}}

%Typing Judgement
\newcommand{\tyb}[3]{#1:_{#3}#2}
\newcommand{\ty}[3]{\vdash\tyb{#1}{#2}{#3}}

\newcommand{\tyx}[2]{#1:#2}
\newcommand{\kd}[2]{\vdash#1 \mathrel{\dblcolon} #2}
\newcommand{\primctx}{\Upsilon}

%Pattern Typing
%% \newcommand{\pty}[2]{\vdash#1\gg_{\upsilon}#2}
%% \newcommand{\ptye}[3]{\vdash#1\gg_{\varepsilon}^{#3}#2}
\newcommand{\pty}[2]{\vdash#1\gg#2}
\newcommand{\ptye}[3]{\vdash#1\mathrel{\gg_{#3}}#2}

%Shortcuts
\newcommand{\al}{\alpha}
\newcommand{\wf}[1]{\vdash #1~\kw{wf}}

%Right-Left
\renewcommand{\l}[1]{\mbox{L}(#1)}
\renewcommand{\ll}[1]{\mbox{L}(\t \lift{#1})}
\newcommand{\rr}[1]{\mbox{R}(\t \lift{#1})}

\renewcommand{\r}[1]{\mbox{R}(#1)}
\newcommand{\extr}[2]{\kw{extract}(#1)=#2}
\renewcommand{\merge}[2]{\kw{merge}(#1,#2)}
%\newcommand{\lift}[1]{\widehat{ \kw{keep}} (#1)}
\newcommand{\liftPat}[1]{\ulcorner#1\urcorner}
\newcommand{\lift}[1]{\ulcorner#1\urcorner}
\newcommand{\vp}{v\!\!v}
\newcommand{\ep}{e\!e}

%Logical relation
\newcommand{\e}{\varepsilon}
\newcommand{\lr}[1]{\llbracket#1\rrbracket_{v}}
\newcommand{\inlr}[1]{\in~\lr{#1}}
\newcommand{\lre}[2]{\llbracket#1\rrbracket_{\varepsilon}^{#2}}
\newcommand{\inlre}[2]{\in~\lre{#1}{#2}}
\newcommand{\inlres}[3]{\in~\lre{#1}{#3{#2}}}

\newcommand{\relwith}[2]{\{#1~|~#2\}}
\newcommand{\rel}[1]{\{#1\}}
\renewcommand{\d}[1]{\mathcal{D}\llbracket#1\rrbracket}
\newcommand{\dd}[1]{\mathcal{D}\llbracket\Delta\rrbracket}
\newcommand{\gsubst}[1]{\mathcal{G}\llbracket#1\rrbracket}
\newcommand{\inr}[1]{\in \gsubst{#1}}

\newcommand{\maps}[2]{#1\mapsto#2}
\renewcommand{\sin}[2]{#1\in\d{#2}}
\newcommand{\sg}{\upvarsigma\Gamma}
\renewcommand{\t}{\uptheta}

\newcommand{\lreq}[3]{#1=\l{#3}~\wedge~ #2=\r{#3}}
\newcommand{\satis}[2]{#1 \models #2}
\newenvironment{nstabbing}
  {\setlength{\topsep}{0pt}%
   \setlength{\partopsep}{0pt}%
   \tabbing}
  {\endtabbing} 

%Pierre-L�o was here
\newcommand{\myinf}[3]{#1;#2 #3 }
\newcommand{\tu}[5]{\uparrow \tyb{#1}{#2}{#3} \Rightarrow [#4]. #5}
\newcommand{\td}[5]{\downarrow \tyb{#1}{#2}{#3} \Rightarrow [#4]. #5}
\newcommand{\subinf}[3]{#1;#2 \vdash #3}
\newcommand{\issub}[3]{#1 \sqsubseteq #2 \Rightarrow #3}
\newcommand{\unify}[2]{#1 \sqcup #2}
\newcommand{\mykinded}[2]{#1 \overset{#2}{\mathrel{::}} K}
\newcommand{\mykind}[2]{#1 \mathrel{::} #2}
\newcommand{\mykindedlist}[2]{#1 \overset{#2}{\mathrel{::}} [K]}
\newcommand{\mykindedbis}[2]{#1 \overset{#2}{\mathrel{::}} K'}


%APPENDIX
\newcommand{\case}[1]{\item \textbf{#1}\newline}
%\newcommand{\case}[1]{\noindent \textbf{#1}\newline}
\newcommand{\caseM}[1]{\item \textbf{Case} $#1$\newline}

\newcommand{\partline}{\begin{center}---------------------------\end{center}}
\newcommand{\blog}{\kw{log}_2}
\newcommand{\bmin}{\kw{min}}

%%% Local Variables: 
%%% mode: plain-tex
%%% TeX-master: "main"
%%% End: 
 

\newtheorem{theorem}{Theorem}
\newtheorem{lemma}[theorem]{Lemma}


\setlength{\bibsep}{0pt plus 0.5ex}





\newenvironment{mathprooftree}
  {\varwidth{.9\textwidth}\centering\leavevmode}
  {\DisplayProof\endvarwidth}


\newcommand{\qm}{\overline{Q}\overline{m}}

\newcommand{\myparagraph}[1]{\paragraph{#1}\mbox{}\\}
\newcommand\myeq{\stackrel{\mathclap{\normalfont\mbox{\scriptsize{def}}}}{=}}

\title{Bases formelles du TAL  \\ DM sur les $\epsilon$-transitions }

\author{Pierre-Léo Bégay}
\date{À me rendre le 6 mars 2020}
\theoremstyle{definition}
\newtheorem{exmp}{Exemple}
   

\begin{document}
 
\maketitle
\pagestyle{empty} %
\thispagestyle{empty}

\newpage
%% Attention: pas plus d'un recto-verso!
% Ne conservez pas les questions

\section{$\epsilon$-transitions}

\subsection{Définitions}

On donne parfois la définition suivante d'un AFND : 


\[
A =  \big \langle Q,\Sigma,q_0,F,\delta \big \rangle
\]

\begin{itemize}
\item[] $Q$ ensemble fini d'états
\item[] $\Sigma$ l'alphabet (ensemble de lettres)
\item[] $q_0$ l'état initial
\item[] $F \subseteq Q$, les états terminaux
\item[] $\delta$ fonction de $(Q \times (\Sigma \cup \{\epsilon\}))$ dans $2^Q$ 
\end{itemize}

Par rapport à la définition du cours, on revient à un seul état initial et qu'on permet d'étiqueter des transitions par $\epsilon$. Ces transitions, appelées $\epsilon$-transitions, sont \textit{gratuites}, par contraste avec les transitions normales qui \textit{consomment} une lettre chaque fois qu'on les emprunte. La notion d'acceptation est sinon la même que pour les AFND qu'on a déjà vus. 

\subsection{Exemples}



\begin{figure}[!h]
\centering

\begin{tikzpicture}[->,>=stealth',shorten >=1pt,auto,node distance=2.5cm,
                    semithick]
  \tikzstyle{every state}=[fill=white,text=black]
  \tikzstyle{place}=[rectangle,draw=black,fill=white, minimum size=7mm]


  \node[initial, state] (S)                    {$q_1$};
  \node[state,accepting] (A)      [right of=S]                {$q_2$};

  \path (S) edge      []        node {$\epsilon$} (A);
\end{tikzpicture}
\caption{Automate $A_1$}
\end{figure}

Dans l'automate $A_1$, aucune transition par lettre n'est possible, ce qui empêche d'accepter tout mot autre que le mot vide. Ce dernier est cependant reconnu car on peut emprunter \textit{gratuitement} l'unique transition et atterrir dans un état terminal.


\begin{figure}[!h]
\centering

\begin{tikzpicture}[->,>=stealth',shorten >=1pt,auto,node distance=2.5cm,
                    semithick]
  \tikzstyle{every state}=[fill=white,text=black]
  \tikzstyle{place}=[rectangle,draw=black,fill=white, minimum size=7mm]


  \node[initial, state] (S)                    {$q_1$};
  \node[state,accepting] (A)      [right of=S]                {$q_2$};

  \path (S) edge[loop above]              node {$a$} (S)
(S) edge      []        node {$\epsilon$} (A)
 (A) edge[loop above]              node {$b$} (A);
\end{tikzpicture}
\caption{Automate $A_2$}
\end{figure}
\newpage
L'automate $A_2$ reconnait quant à lui le langage $a^*b^*$. En effet, on peut boucler avec des $a$ sur $q_1$ puis, une fois qu'on a fini, on passe gratuitement à $q_2$ (sans consommer de $a$ ou de $b$) où on peut boucler avec des $b$ jusqu'à avoir fini le mot.

\begin{figure}[!ht]
\centering

\begin{tikzpicture}[->,>=stealth',shorten >=1pt,auto,node distance=2.5cm,
                    semithick]
  \tikzstyle{every state}=[fill=white,text=black]
  \tikzstyle{place}=[rectangle,draw=black,fill=white, minimum size=7mm]


  \node[initial, state] (S)                    {$q_1$};
  \node[state] (A)      [above right of=S]                {$q_2$};
  \node[state] (B)      [right of=A]                {$q_3$};
  \node[state] (C)      [right of=B]                {$q_4$};
    \node[state] (D)      [right of=C]                {$q_5$};
  \node[state] (E)      [below right of=S]                {$q_6$};
  \node[state] (F)      [right of=E]                {$q_7$};
    \node[state] (G)      [right of=F]                {$q_8$};    \node[state] (H)      [right of=G]                {$q_9$};
        \node[state,accepting] (I)      [above right of=H]                {$q_{10}$};
  
  \path (S) edge[loop above]              node {$a,b$} (S)
 (I) edge[loop above]              node {$a,b$} (I)
(S) edge      []        node {$\epsilon$} (A)
(A) edge      []        node {$a$} (B)
(B) edge      []        node {$b$} (C)
(C) edge      []        node {$a$} (D)
(D) edge      []        node {$\epsilon$} (I)
(S) edge      []        node {$\epsilon$} (E)
(E) edge      []        node {$b$} (F)
(F) edge      []        node {$a$} (G)
(G) edge      []        node {$b$} (H)
(H) edge      []        node {$\epsilon$} (I)
(C) edge      [bend left]        node {$\epsilon$} (F)
(G) edge      [bend left]        node {$\epsilon$} (B);
\end{tikzpicture}
\caption{Automate $A_3$}
\end{figure}

Enfin, l'automate $A_3$, proche d'un qu'on a vu en cours, reconnaît quant à lui le langage $\Sigma^*aba\Sigma^* + \Sigma^*bab\Sigma^*$ (les deux $\epsilon$-transitions en croix ne permettent pas d'accepter plus de mots).



\section{Lecture d'automates avec $\epsilon$-transitions}


Décrivez les langages reconnus par les automates $A_4$, $A_5$ et $A_6$ à l'aide d'une expression rationelle. Essayez de justifier, au moins informellement, votre réponse. 

\begin{figure}[!h]
\centering

\begin{tikzpicture}[->,>=stealth',shorten >=1pt,auto,node distance=2.5cm,
                    semithick]
  \tikzstyle{every state}=[fill=white,text=black]
  \tikzstyle{place}=[rectangle,draw=black,fill=white, minimum size=7mm]


  \node[initial, state] (S)                    {$q_1$};
  \node[state] (A)      [above right of=S]                {$q_2$};
  \node[state] (B)      [right of=A]                {$q_3$};
  \node[state] (C)      [below right of=S]                {$q_4$};
    \node[state] (D)      [right of=C]                {$q_5$};
  \node[state] (E)      [above right of=D]                {$q_6$};
  \node[state,accepting] (F)      [right of=E]                {$q_7$};
    
  
  \path %(I) edge[loop above]              node {$a,b$} (I)
(S) edge      []        node {$\epsilon$} (A)
(A) edge      []        node {$a$} (B)
(B) edge      []        node {$\epsilon$} (S)
(S) edge      []        node {$\epsilon$} (C)
(C) edge      []        node {$b$} (D)
(D) edge      []        node {$\epsilon$} (S)
(S) edge      []        node {$\epsilon$} (E)
(E) edge      []        node {$c$} (F);
\end{tikzpicture}
\caption{Automate $A_4$}
\end{figure}



\begin{figure}[!h]
\centering

\begin{tikzpicture}[->,>=stealth',shorten >=1pt,auto,node distance=2.5cm,
                    semithick]
  \tikzstyle{every state}=[fill=white,text=black]
  \tikzstyle{place}=[rectangle,draw=black,fill=white, minimum size=7mm]


  \node[initial, state] (S)                    {$q_1$};
  \node[state] (A)      [above of=S]                {$q_2$};
  \node[state] (B)      [above of=A]                {$q_3$};
  \node[state] (C)      [below left of=B]                {$q_4$};
    \node[state] (D)      [right of=S]                {$q_5$};
  \node[state] (E)      [right of=D]                {$q_6$};
  \node[state] (F)      [right of=E]                {$q_7$};
    \node[state,accepting] (G)      [right of=F]                {$q_8$};    
    
  
  \path %(I) edge[loop above]              node {$a,b$} (I)
(S) edge      []        node {$\epsilon$} (A)
(A) edge      []        node {$a$} (B)
(B) edge      []        node {$b$} (C)
(C) edge      []        node {$\epsilon$} (S)
(S) edge      []        node {$\epsilon$} (D)
(D) edge      []        node {$a$} (E)
(E) edge      []        node {$a$} (F)
(F) edge      []        node {$\epsilon$} (G)
(S) edge      [bend left]        node {$\epsilon$} (G);
\end{tikzpicture}
\caption{Automate $A_5$}
\end{figure}

\newpage

\begin{figure}[!h]
\centering

\begin{tikzpicture}[->,>=stealth',shorten >=1pt,auto,node distance=2.5cm,
                    semithick]
  \tikzstyle{every state}=[fill=white,text=black]
  \tikzstyle{place}=[rectangle,draw=black,fill=white, minimum size=7mm]


  \node[initial, state] (S)                    {$q_1$};
  \node[state] (A)      [right of=S]                {$q_2$};
  \node[state] (B)      [right of=A]                {$q_3$};
  \node[state] (C)      [right of=B]                {$q_4$};
    \node[state] (D)      [right of=C]                {$q_5$};
  \node[state,accepting] (E)      [right of=D]                {$q_6$};

    
  
  \path %(I) edge[loop above]              node {$a,b$} (I)
(S) edge      []        node {$a$} (A)
(A) edge      []        node {$b$} (B)
(S) edge      [bend left]        node {$\epsilon$} (B)
(B) edge      [bend left]        node {$\epsilon$} (S)
(B) edge      [bend left]        node {$\epsilon$} (D)
(D) edge      [bend left]        node {$\epsilon$} (B)
(S) edge      [bend left]        node {$\epsilon$} (D)
(D) edge      [bend left]        node {$\epsilon$} (S)
(B) edge      []        node {$a$} (C)
(C) edge      []        node {$c$} (D)
(D) edge      []        node {$c$} (E);
\end{tikzpicture}
\caption{Automate $A_6$}
\end{figure}

\section{Elimination d'$\epsilon$-transitions}

On propose une méthode pour éliminer les $\epsilon$-transitions s'appuyant sur la fonction $\epsilon^+$ de type $Q \rightarrow 2^Q$ définie de la façon suivante :

\begin{figure}[!h]
\begin{enumerate}
\item Si $q_j \in \delta(q_i,\epsilon)$, alors $q_j \in \epsilon^+(q_i)$
\item Si $q_j \in \epsilon^+(q_i)$ et $q_k \in \delta(q_j,\epsilon)$, alors $q_k \in \epsilon^+(q_i)$
\end{enumerate}
\caption{Définition de $\epsilon^+$}
\label{eplusdef}
\end{figure}
 
 
A partir d'un automate non-déterministe avec $\epsilon$-transitions  $\big \langle Q,\Sigma,q_0,F,\delta \big \rangle$, on génère un automate non-déterministe équivalent sans $\epsilon$-transitions  $\big \langle Q,\Sigma,q_0,F',\delta' \big \rangle$ avec l'algorithme suivant :

\begin{figure}[!h]
\begin{algorithmic}[1]
\State $F' := F$
\ForAll {$q_i \in Q$}
    \ForAll {$c \in \Sigma$}
        $\delta'(q_i,c) := \delta(q_i,c)$
    \EndFor
\EndFor
\ForAll{$q_i \in Q$ tels que $\epsilon^+(q_i) \neq \emptyset$}
	\ForAll{$q_j \in \epsilon^+(q_i)$}
		\ForAll{$c \in \Sigma$ et $q_r \in Q$ tels que $q_r \in \delta(q_j,c)$}
			\State $\delta'(q_i,c) := \delta'(q_i,c) \cup \{q_r\}$
		\EndFor
		\If {$q_j \in F$}
			\State $F' := F' \cup \{q_i\}$
		\EndIf
	\EndFor
\EndFor
\end{algorithmic}
\caption{Algorithme d'élimination des $\epsilon$-transitions}
\label{algo}
\end{figure}
\vspace{2cm}
\paragraph*{Question 1} Pour chaque automate ($A_1$ à $A_6$), calculez la fonction $\epsilon^+$ en vous servant de la définition en figure \ref{eplusdef}. Vous devriez donner l'image de chaque état, par exemple $\epsilon^+(q_1) = \{q_1,q_2\}$, $\epsilon^+(q_2) = \emptyset$ etc. 

\paragraph*{Question 2} Pour chaque automate ($A_1$ à $A_6$), appliquez l'algorithme de la figure \ref{algo}. Vous pouvez dessiner le résultat. Pas besoin de détailler tous les calculs. 

\paragraph*{Question 3} Essayez d'expliquer en français la fonction $\epsilon^+$ et l'algorithme comme si vous vouliez me convaincre qu'ils font correctement leur boulot (ce qui est le cas)\footnote{Notez que rien n'est gratuit dans l'algorithme et que chaque \textit{morceau} a un sens. Vous devriez donc tout mentionner.}. Vous pouvez vous aider d'exemples, soit tirés de la question précédente, soit originaux.

\section{Formalisation}

\paragraph*{Question 4} Donnez une formalisation de l'acceptation d'un mot dans le contexte des AFND avec $\epsilon$-transtion en adaptant la définition de $\delta^*$ donnée dans le cours.

\paragraph*{Question bonus} Les AFND avec $\epsilon$-transitions sont-ils plus ou moins expressifs\footnote{cad. permettent-ils de décrire plus ou moins de langages ?} que ceux vus en cours, où on pouvait avoir plusieurs états initiaux et plusieurs transitions "concurrentes" ?


\section{Propriétés de clôture}

\paragraph*{Question 5} Etant donnés des AFND (version $\epsilon$) représentant deux expressions rationnelles quelconques $e_1$ et $e_2$, expliquez, en vous aidant de schémas, comment construire des automates reconnaissant les expressions $e_1 + e_2$, $e_1.e_2$ et $e_1^*$.

\paragraph*{Question bonus} Répondre à la question précédente en utilisant la formalisation des automates.

\bibliographystyle{abbrv}
\bibliography{main}  % main.bib is disused
%\end{multicols}



\end{document} 
  


 
