\chapter{Grammaires formelles et hiérarchie de Chomsky}
\label{grammaires}


\section{Principe général}

Une \textbf{grammaire formelle} est une série de \textbf{règles} permettant de générer des mots. Ces règles utilisent des \textbf{symboles} dits \textbf{terminaux} (les lettres "normales", par convention en minuscules), et d'autres dits \textbf{non-terminaux} (normalement dénotés par des lettres majuscules). Un de ces symboles non-terminaux, appelé \textbf{axiome}, indique le début de toute génération de mot. On présente d'abord le fonctionnement des grammaires et les concepts de base à l'aide de quelques exemples. 

\begin{example}
On présente la grammaire dénotée par ces deux règles :

\[
\begin{cases}
S \rightarrow \epsilon \\
S \rightarrow abS 
\end{cases}
\]

Cette grammaire contient un seul symbole non-terminal, S. Il s'agit donc automatiquement de l'axiome. Le symbole S peut se transformer en $\epsilon$ (première règle) ou en $abS$ (deuxième règle). Cette grammaire génère l'ensemble des mots composés uniquement de symboles terminaux ($a$ et $b$) qu'on peut obtenir en partant de l'axiome S et en appliquant autant de fois qu'on veut les règles données. Dans ce qui suit, on écrira $\rightarrow_1$ pour une application de la première règle et $\rightarrow_2$ pour la seconde.

On peut par exemple obtenir le mot $ab$ de la façon suivante :

\[
S \rightarrow_2 abS \rightarrow_1 ab
\]

Dans cette suite de transformation, appelée \textbf{dérivation}, on remplace d'abord l'axiome par $abS$ à l'aide de la deuxième règle. Puisque $abS$ contient le facteur $S$, on peut utiliser \textit{localement} la deuxième règle pour faire disparaître ce S. Dans ce cas, ce qu'il y avait autour du S, le \textbf{contexte}, reste inchangé.

On peut également obtenir le mot $abab$ :

\[
S \rightarrow_2 \textcolor{blue}{ab}S \rightarrow_2 \textcolor{blue}{ab}\textcolor{red}{ab}S \rightarrow_1 \textcolor{blue}{ab}\textcolor{red}{ab}
\]

ou ababab :


\[
S \rightarrow_2 \textcolor{blue}{ab}S \rightarrow_2 \textcolor{blue}{ab}\textcolor{red}{ab}S \rightarrow_2 \textcolor{blue}{ab}\textcolor{red}{ab}\textcolor{green}{ab}S \rightarrow_1  \textcolor{blue}{ab}\textcolor{red}{ab}\textcolor{green}{ab}
\]

Il n'est pas obligatoire d'utiliser toutes les règles d'une grammaire. On peut donc générer le mot vide :

\[
S \rightarrow_1 \epsilon
\]

On se rend compte assez vite que la grammaire \textbf{engendre} le langage $(ab)^*$.

\end{example}


\begin{example}
\label{LRgram}
Un symbole non-terminal peut générer d'autres symboles non-terminaux, et même plusieurs en même temps. De plus, un non-terminal peut générer un autre mot que juste $\epsilon$. Ces deux points sont illustrés par la grammaire suivante :

\[
\begin{cases}
S \rightarrow LaaR \\
L \rightarrow Lb \\
L \rightarrow ab \\
R \rightarrow bR \\
R \rightarrow ba
\end{cases}
\]

Quelques exemples de dérivations dans cette grammaire :

\[
 S \rightarrow_1 LaaR \rightarrow_2 LbaaR \rightarrow_2 LbbaaR \rightarrow_4 LbbaabR \rightarrow_3 abbbaabR \rightarrow_5 abbbaabba
\]

\[
 S \rightarrow_1 LaaR \rightarrow_3 abaaR \rightarrow_5 abaaba
 \]
 
 On note $\rightarrow_i^j$ $j$ utilisations de la règle numéro $i$, comme dans la dérivation suivante :
 
 \[
 S \rightarrow_1 LaaR \rightarrow_2^5 LbbbbbaaR \rightarrow_3 abbbbbbaaR \rightarrow_4^3 abbbbbbaabbbR \rightarrow_5 abbbbbbaabbbba
 \]

\end{example}

\begin{exercice}
Quel est le langage engendré par la grammaire de l'exemple \ref{LRgram} ?
\end{exercice}

\begin{exercice}
\label{grammab}
Quel est le langage reconnu par la grammaire suivante ?

\[
\begin{cases}
S \rightarrow A \\
S \rightarrow B \\
A \rightarrow aA \\
A \rightarrow \epsilon \\
B \rightarrow bB \\
B \rightarrow \epsilon
\end{cases}
\]

\end{exercice}

\begin{exercice}
\label{grammsigma}
Quel est le langage reconnu par la grammaire suivante ?

\[
\begin{cases}
S \rightarrow aS \\
S \rightarrow bS \\
S \rightarrow \epsilon 
\end{cases}
\]

\end{exercice}

\begin{exercice}
Donner l'ensemble des mots qui admettent deux dérivations (ie. peuvent être construits de plusieurs façons différentes) dans la grammaire de l'exercice \ref{grammab}. Même question pour celle de l'exercice \ref{grammsigma}.
\end{exercice}

\begin{example}
On a jusqu'ici seulement vu des exemples de grammaires avec un seul non-terminal à gauche des flèches de réécriture. Il est cependant possible de préciser le contexte dans lequel les réécritures doivent se faire, comme dans la grammaire suivante :

\[
\begin{cases}
S \rightarrow SABC \\
S \rightarrow \epsilon \\
AB \rightarrow BA \\
BA \rightarrow AB \\
AC \rightarrow CA \\
CA \rightarrow AC \\
BC \rightarrow CB \\
CB \rightarrow BC \\
A \rightarrow a \\
B \rightarrow b \\
C \rightarrow c
\end{cases}
\]

Toute dérivation dans cette grammaire fonctionne de la façon suivante : on commence par utiliser la première règle $n$ fois pour produire $S(ABC)^n$. Ensuite, on utilise la deuxième règle pour faire disparaître S et se retrouver avec $(ABC)^n$. Les règles 3 à 8 permettent ensuite de mélanger les symboles non-terminaux comme on l'entend. Une fois qu'ils ont été disposés de la façon voulue, on les transforme en $a$, $b$ et $c$ avec les règles 9, 10 et 11.

Le langage engendré par cette grammaire est donc celui des mots contenant autant de $a$ que de $b$ et de $c$.
\end{example}

\paragraph{Remarque} Dans l'exemple ci-dessus, on découpe toute dérivation en 4 étapes : utilisation de la première règle, puis de la seconde, puis des 3 à 8, et enfin des 9 à 11. Cette présentation nous semble améliorer la compréhension de l'exemple et du langage engendré, mais techniquement, rien n'empêche de mélanger les étapes, comme dans la dérivation suivante :

 \[
 S \rightarrow_1 SABC \rightarrow_9 SaBC \rightarrow_7 SaCB \rightarrow_{11} SaCb \rightarrow_1 SABCaCb \rightarrow_2 ABCaCB \rightarrow_{9,10,11}^5 abcacb 
 \]

 Où $\rightarrow_{9,10,11}^5$ indique 5 utilisations de règles parmi 9, 10 et 11.
 
 