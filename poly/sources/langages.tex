
\chapter{Langages}
\label{langages}

On définit d'abord la notion de mot, nécessaire à celle de langage. On verra ensuite comment décrire des langages à l'aide de notations ensemblistes, révisant ces dernières par la même occasion.

\section{Mots}

\begin{definition}{Mot}{}Un \textbf{mot} est une suite de lettres tirées d'un alphabet donné. L'ensemble des mots sur un alphabet $\Sigma$ est noté $\Sigma^*$.
\end{definition}

\begin{example}
Etant donné l'alphabet $\Sigma = \{a,b,c\}$, on peut construire une infinité de mots parmi lesquels 

\begin{itemize}
    \item $abc$
    \item $aab$
    \item $cc$
    \item $abcabcacbacbacbacbabcabcabcabcabcabcabc$
    \item $a$
\end{itemize}

\end{example}


\paragraph{Remarque} On va s'intéresser ici à des langages et mots complètement abstraits, en général composés uniquement de $a$, $b$ et $c$.

\begin{definition}{Mot vide}{}
Une suite de lettres peut être de longueur zéro, formant alors \textbf{le mot vide}. Quel que soit l'alphabet, ce dernier sera noté $\epsilon$.
\end{definition}

\begin{definition}{Concaténation}{}
\label{concat}
L'opération de \textbf{concaténation}, notée $.$, consiste tout simplement à "coller" deux mots.
\end{definition}

\begin{example}{Quelques concaténations :}

\begin{itemize}
    \item $ab.c = abc$
    \item $ab.ba = abba$
\end{itemize}
\end{example}

 De plus, pour tout mot $w$, 

\[
    w.\epsilon = \epsilon.w = w
\]

\paragraph{Remarque} Les algébristes enthousiastes remarqueront que $(\Sigma^*,.,\epsilon)$ forme un monoïde libre de base $\Sigma$


\begin{definition}{Longueur d'un mot}{}
Etant donné un mot $w$, on note sa \textbf{longueur} $|w|$.
\end{definition}

\begin{example}
Tout naturellement,
\begin{itemize}
    \item $|abc| = 3$
    \item $|abba| = 4$
    \item $|c| = 1$
    \item $|\epsilon| = 0$
\end{itemize}
\end{example}

\begin{definition}{Principe d'induction sur un mot}{}
Etant donnée une propriété $P$ sur les mots. Si on a\\

\begin{enumerate}
\item $P(\epsilon)$ (cad. que $P$ est vraie pour le mot vide)
\item $\forall w, \forall c \in \Sigma, (P(w) \rightarrow P(c.w))$ (cad. que si $P$ est vraie pour un mot, alors elle reste vraie si on rajoute n'importe quelle lettre à gauche du mot)\\
\end{enumerate}

Alors la propriété $P$ est vraie pour tout mot $w$.
\end{definition}

\paragraph*{Remarque} Est également valide le principe d'induction où, dans le cas récursif, la lettre est rajoutée à droite du mot plutôt qu'à sa gauche.


On va s'entraîner à utiliser ce principe d'induction en prouvant deux lemmes qui n'en nécessitaient sans doute pas tant :

\begin{lemma}
$\forall w \in \Sigma^*, |w| \geq 0$, cad. que tout mot a une longueur positive.
\end{lemma}
\begin{proof}
On procède par induction sur $w$.

Dans le cas de base, $w = \epsilon$. On a donc $|w| = |\epsilon| = 0 \geq 0$.

Dans le cas récursif, $w = c.w'$ avec $c \in \Sigma$ et on suppose $|w'| \geq 0$. On a $|c.w'| = 1 + |w'| \geq |w'| \geq 0$.
\end{proof}


\begin{lemma}
Etant donnés deux mots $w_1$ et $w_2$, $|w_1.w_2| = |w_1| + |w_2|$. 
\end{lemma}

\begin{proof}
On procède par induction sur $w_1$.

Dans le cas de base, $w_1 = \epsilon$. On a donc $|w_1.w_2| = |\epsilon.w_2| = |w_2| = 0 + |w_2| = |w_1| + |w_2|$.

Dans le cas récursif, $w_1 = c.w_1'$ avec $c \in \Sigma$ et on suppose $|w_1'.w_2| = |w_1'| + |w_2|$. On a
\begin{tabular}{cll}
&&\\
& $|w_1.w_2|$ & \\
$=$ & $|c.w_1'.w_2|$ & par définition de $w_1$ \\
$=$&  $1 + |w_1'.w_2|$ & par définition de $|.|$ \\
$=$& $1 + (|w_1'| + |w_2|)$ & par hypothèse d'induction \\
$=$& $(1 + |w_1'|) + |w_2|$ & par associativité de l'addition \\
$=$& $|c.w_1'| + |w_2|$ & par définition de $|.|$ \\
$=$& $|w_1| + |w_2|$ & par définition de $w_1$
\end{tabular}

On a donc bien nos deux conditions pour le raisonnement par induction.
\end{proof}


\begin{definition}{Nombre d'occurrences d'une lettre}{}
Etant donné un mot $w$ et une lettre $a$, on note $\mathbf{|w|_a}$ le nombre de $a$ dans $w$.
\end{definition}

\begin{example}
On a 
\begin{itemize}
    \item $|abc|_a = 1$
    \item $|abba|_b = 2$
    \item $|c|_a = 0$
    \item $|\epsilon|_a = 0$
\end{itemize}
\end{example}

\begin{definition}{Préfixe}{}
Un mot $p$ est un \textbf{préfixe} du mot $w$ ssi $\exists v, w = p.v$, cad. ssi $w$ \textit{commence} par $p$.
\end{definition}

\begin{definition}{Suffixe}{}
Un mot $s$ est un \textbf{suffixe} du mot $w$ ssi $\exists v, w = v.s$, cad. ssi $w$ \textit{finit} par $p$.
\end{definition}

\begin{example}
Le mot $abba$ admet comme préfixes $\epsilon$, $a$, $ab$, $abb$ et $abba$. Ses suffixes sont, quant à eux, $\epsilon$, $a$, $ba$, $bba$ et $abba$.
\end{example}

\begin{lemma}
$\forall w \in \Sigma^*, \epsilon$ et $w$ sont des préfixes de $w$
\end{lemma}

\begin{proof}
Pour $\epsilon$, il suffit de prendre $v = w$. A l'inverse, en prenant $v = \epsilon$, on voit que $w$ est son propre préfixe.
\end{proof}

\begin{lemma}
$\forall w \in \Sigma^*, \epsilon$ et $w$ sont des suffixes de $w$
\end{lemma}

\begin{proof}
Analogue au lemme précédent.
\end{proof}


\begin{exercice}\label{expref}
Combien de préfixes et suffixes admet un mot $w$ quelconque ?
\end{exercice}

\begin{definition}{Facteur}{}
Un mot $f$ est un \textbf{facteur} du mot $w$ ssi $\exists v_1 v_2, w = v_1.f.v_2$, cad. ssi $f$ apparaît dans $w$.
\end{definition}

\begin{example}
Les facteurs du mot $abba$ sont $\epsilon$, $a$, $b$, $ab$, $ba$, $abb$, $bba$ et $abba$. 
\end{example}

\begin{lemma}
$\forall w \in \Sigma^*$, $\epsilon$ et $w$ sont des facteurs de $w$.
\end{lemma}

\begin{proof}
Pour $\epsilon$, il suffit de prendre $v_1 = w$ et $v_2 = \epsilon$ (ou l'inverse) et la condition est trivialement vérifiée. Pour $w$, on prend $v_1 = v_2 = \epsilon$.
\end{proof}

\begin{exercice}
Donner l'ensemble des facteurs du mot $abbba$.
\end{exercice}

\begin{exercice} \label{exfact}(*)
Donner la borne la plus basse possible du nombre de facteurs d'un mot $w$. Donner un mot d'au moins 3 lettres dont le nombre de facteurs est exactement la borne donnée.
\end{exercice}

\begin{definition}{Sous-mot}{}
Un mot $s$ est un \textbf{sous-mot} du mot $w$ ssi $w = v_0s_0v_1s_1v_2...s_nv_n$ et $s = s_0s_1...s_n$, cad. ssi $w$ est "$s$ avec (potentiellement) des lettres en plus".
\end{definition}

\begin{example}\label{ex5} On \underline{souligne} les lettres originellement présentes dans le sous-mot :
\begin{itemize}
   \item $ab$ est un sous-mot de $b\underline{a}a\underline{b}$, qu'on pourrait aussi voir comme $ba\underline{ab}$
   \item $abba$ est un sous-mot de $ba\underline{a}abaa\underline{bb}a\underline{a}$.
   \item $ba$ \underline{n}'est \underline{pas} un sous-mot de $aaabbb$ (l'ordre du sous-mot doit être préservé dans le mot) 
\end{itemize}
\end{example}


\begin{lemma}
$\forall w \in \Sigma^*$, $\epsilon$ et $w$ sont des sous-mots de $w$.
\end{lemma}

\begin{proof}
Pour $\epsilon$, il suffit de prendre $n = 0$, $s_0 = \epsilon$, $v_0 = w$ et $v_1 = \epsilon$ (ou l'inverse) et la condition est trivialement vérifiée. Pour $w$, on prend $n = 0$, $s_0 = w$ et $v_0 = v_1 = \epsilon$.
\end{proof}

\begin{exercice}
Montrer que tout facteur d'un mot en est également un sous-mot. A l'inverse, montrer qu'un sous-mot n'est pas forcément un facteur. 
\end{exercice}

\begin{exercice}
Donner toutes les façons de voir $abba$ comme sous-mot de $baaabaabbaa$ (cf. exemple \ref{ex5}).
\end{exercice}

\begin{exercice}
Donner l'ensemble des sous-mots de $abba$
\end{exercice}

\begin{exercice}\label{exssmot} (*)
Donner la borne la plus basse possible du nombre de sous-mots d'un mot $w$. Donner un mot dont le nombre de sous-mots est exactement la borne donnée.
\end{exercice}

\begin{exercice} (*) Dans l'exercice \ref{expref}, on demande le nombre exact de préfixes et suffixes d'un mot, alors que dans les exercices \ref{exfact} et \ref{exssmot}, on demande une borne, pourquoi ?
\end{exercice}

\section{Langage}


Un langage, c'est tout simplement un ensemble de mots. On distingue d'entrée deux langages un peu spéciaux, pour ne pas dire pathologiques.

\begin{definition}{Langage plein}{}
Etant donné un alphabet $\Sigma$, on note $\mathbf{\Sigma^*}$ l'ensemble de tous les mots formés à partir de $\Sigma$.
\end{definition}


\begin{definition}{Ensemble/Langage vide}{}
On note $\emptyset$ l'ensemble vide, ou langage vide en ce qui nous concerne, qui se caractèrise comme ne contenant aucun élément.
\end{definition}

\paragraph{Remarque} Ne surtout pas confondre $\emptyset$ et $\{\epsilon\}$. Le premier est un ensemble vide, contenant donc \textbf{0} élément, trandis que le second contient \textbf{1} élément, le mot ide. 

Ces deux ensembles sont bien gentils, mais un peu radicaux. La question maintenant est donc de savoir comment on définit et parle de langages précis et plus "intermédiaires". En tout généralité, les ensembles peuvent être définis de façon \textbf{extensionnelle} ou \textbf{intentionnelle}. 

\begin{definition}{Définition extensionnelle d'un ensemble}{}
On \textbf{définit extensionnellement un ensemble} en en donnant la liste des éléments. L'ensemble vide se note quant à lui $\emptyset$.
\end{definition}

\begin{example} On définit par exemple l'ensemble (sans intérêt) suivant :
\[
    A = \{b, aca, abba\}
\]
\end{example}

Les définitions extensionnelles ont le mérite d'être pour le moins simples, mais pas super pratiques quand il s'agit de définir des ensembles avec un nombre infini d'éléments, comme l'ensemble des mots de longueur pair. Les définitions intentionnelles permettent elles de gérer ça, en utilisant des prédicats.

\begin{definition}{Prédicat}{}
Un prédicat est une propriété, sur un ou plusieurs\footnote{Ou zéro, mais bon} éléments. On note les prédicats comme des fonctions, par exemple $P(x)$ un prédicat unaire et $Q(x,y)$ un prédicat binaire.
\end{definition}

\begin{example}
On pose $P(x) \equiv$ x est de longueur paire. On a alors $P(ab)$ ou $P(\epsilon)$, mais pas $P(a)$ (ce qu'on note $\neg P(a)$). 
\end{example}

\begin{example}
\label{predL}
On pose $L(x,y) \equiv$ $|x| = |y|$. On a alors $L(ab,ba)$, $L(aaaa,bbbb)$ et $\neg L(aba,ba)$.
\end{example}

\begin{exercice}
Donner l'ensemble des mots $w$ tels que $L(w,\epsilon)$. 
\end{exercice}

\begin{exercice}
Soit $\Sigma = \{a,b\}$, donner l'ensemble des mots $w$ tels que $L(aba,w)$. 
\end{exercice}

\begin{definition}{Définition intensionnelle d'un ensemble}{}
On \textbf{définit intensionnellement un ensemble} à l'aide d'une propriété que tous ses éléments satisfont. Étant donnés une propriété $Q(x)$ (par ex. une formule logique) et un ensemble $A$, on note $\{x \in A~|~Q(x)\}$ l'ensemble des éléments de $A$ qui satisfont $P$. Si l'ensemble $A$ est évident dans le contexte, on s'abstiendera de le préciser.
\end{definition}

\begin{example}
On peut définir l'ensemble des mots de longueur paire $\{x \in \Sigma^*~|~P(x)\}$, ou plus simplement $\{x ~|~P(x)\}$.
\end{example}

\begin{exercice}
Donner une définition de l'ensemble des mots de longueur 3 en utilisant le prédicat $L$ défini dans l'exemple \ref{predL}. 
\end{exercice}

Si les définitions intentionnelles permettent, contrairement aux extensionnelles, de dénoter des ensembles contenant une infinité de mots, elles sont avant tout un outil théorique. En effet, une propriété comme "le mot $w$ a une longueur paire" ne dit rien à un ordinateur en soi, et doit donc être définie formellement. Se pose alors la question d'un langage pour les propriétés.

Si plusieurs logiques équipées des bonnes primitives peuvent être utilisées, les traductions sont rarement très agréables. Certaines propriétés nécessitent en effet de ruser contre le langage, voire sont impossibles à formaliser dans certaines logiques. Il existe heureusement un outil qui va nous aider, avec le premier problème du moins.

\chapter{Expressions régulières}
\label{regex}
Les expressions régulières permettent définir de façon finie - et relativement intuitive - "la forme" des mots d'un langage, potentiellement infini. On en présentera d'abord le lexique et l'idée générale à l'aide d'exemples, puis on en définira formellement la syntaxe et la sémantique.

\section{Lexique et idée générale}

Une expression rationnelle (ou \textit{regex}, pour \textit{regular expression}\footnote{On se trompera sans doute souvent en parlant d'"expression régulière"}) est, en gros, une \textit{forme de mot}, écrite à l'aide de lettres et des symboles $.$, $*$ et $+$.

\subsection{Les lettres et $\epsilon$, la base}

Les regex sont construites récursivement, en partant bien sûr des cas de base. Étant donné un alphabet $\Sigma$, ces derniers sont les différentes lettres de $\Sigma$, ainsi que $\epsilon$. Ces regex dénotent chacune un seul mot, la lettre utilisée dans le premier cas, et le mot vide dans le second.


\begin{example}
La \textit{regex} $a$ dénote le langage $\{a\}$.
\end{example}


\subsection{$.$, la concaténation}

On peut heureusement concaténer des regex en utilisant à nouveau le symbole $.$\footnote{avec cependant un type légèrement différent, comme on le verra en \ref{resem}}. La concaténation de deux expressions rationnelles $e_1$ et $e_2$, notée $e_1.e_2$ donc, dénote l'ensemble des mots qui peuvent se décomposer en une première partie "de $e_1$" et une deuxième "de $e_2$".

\paragraph{Remarque} En pratique, on ne notera pas les $.$ dans les regex, mais quelque chose comme $abbc$ devrait en théorie être lu comme $a.b.b.c$


\begin{example}
La \textit{regex} $abca$ dénote l'ensemble $\{abca\}$.
\end{example}


\subsection{*, l'itération}

Le symbole $*$ permet de dire qu'une regex peut être répétée autant de fois que voulu (y compris 0). 

\begin{example}
La regex $ab^*c$ dénote l'ensemble des mots de la forme "un $a$, puis une série (éventuellement vide) de $b$, puis un $c$", c'est-à-dire $\{ac, abc, abbc, abbbc, ...\}$
\end{example}

En utilisant des parenthèses, on peut appliquer $*$ à des facteurs entiers :

\begin{example}
La regex $(aa)^*$ dénote l'ensemble des mots de la forme "une série (éventuellement vide) de deux $a$", c'est-à-dire $\{\epsilon, aa, aaaa, aaaaaa, ...\}$, ou encore les mots composés uniquement de $a$ et de longueur paire.
\end{example}

On peut bien sûr utiliser plusieurs $*$ dans une même expression. Dans ce cas, les nombres de "copies" des facteurs concernés ne sont pas liés, comme l'illustrent les examples suivant :


\begin{example}
\label{ex0}
La regex $a^*b^*$ dénote l'ensemble des mots de la forme "Une série (éventuellement vide) de $a$, puis une série (éventuellement vide) de $b$", contenant notamment $\epsilon$, $a$, $b$, $aaab$, $abbbb$, $bb$, $aaa$, $ab$, $aabb$ et $aaabb$
\end{example}

\begin{example}
\label{ex1}
La regex $ab(bab)^*b(ca)^*b$ dénote l'ensemble des mots de la forme "$ab$, puis une série (éventuellement vide) de $bab$, puis un $b$, puis une série (éventuellement vide) de $ca$, puis un $b$", contenant notamment $abbb$, $abbabbb$, $abbcab$ ou $abbabbabbabbcab$.
\end{example}

\begin{exercice}
Donner 5 autres mots appartenant au langage dénotée par l'expression de l'exemple \ref{ex1}. 
\end{exercice}

\begin{exercice}
Pourquoi le changement de formulation dans les exemples \ref{ex0} et \ref{ex1} par rapport aux exemples précédents ("c'est à dire $\{x, y, z ... \}$" qui devient "contenant notamment $x$, $y$ ou $z$") ?
\end{exercice}

On peut même faire encore plus rigolo, en enchâssant les étoiles:


\begin{example}
\label{ex2}
La regex $(a^*b^*)^*$ dénote l'ensemble des mots de la forme "une série (éventuellement vide) de [(une série (éventuellement vide) de a] puis [une série (éventuellement vide) de b]]", contenant notamment $\epsilon$, $abba$; $baba$ ou $abbbabba$. 
\end{example}

\paragraph{Remarque} Certaines regex peuvent générer des mêmes mots de plusieurs façons. Si on prend l'expression de l'exemple \ref{ex2} et le mot $abbbabba$, on peut la voir comme 

\begin{itemize}
    \item $\underbrace{\underbrace{\underbrace{a}_\text{$a^1$}\underbrace{bbb}_\text{$b^3$}}_\text{$a^1b^3$}\underbrace{\underbrace{a}_\text{$a^1$}\underbrace{bb}_\text{$b^2$}}_\text{$a^1b^2$}\underbrace{\underbrace{a}_\text{$a^1$}\underbrace{ }_\text{$b^0$}}_\text{$a^1b^0$}}_\text{$(a^*b^*)^3$}$   
    \item $\underbrace{\underbrace{\underbrace{a}_\text{$a^1$}\underbrace{b}_\text{$b^1$}}_\text{$a^1b^1$}\underbrace{\underbrace{ }_\text{$a^0$}\underbrace{bb}_\text{$b^2$}}_\text{$a^0b^2$}\underbrace{\underbrace{a}_\text{$a^1$}\underbrace{b }_\text{$b^1$}}_\text{$a^1b^1$}\underbrace{\underbrace{ }_\text{$a^0$}\underbrace{b }_\text{$b^1$}}_\text{$a^0b^1$}\underbrace{\underbrace{a}_\text{$a^1$}\underbrace{  }_\text{$b^0$}}_\text{$a^1b^0$}}_\text{$(a^*b^*)^5$}$
   
    \item $\underbrace{\underbrace{\underbrace{a}_\text{$a^1$}\underbrace{b}_\text{$b^1$}}_\text{$a^1b^1$}\underbrace{\underbrace{ }_\text{$a^0$}\underbrace{ }_\text{$b^0$}}_\text{$a^0b^0$}\underbrace{\underbrace{ }_\text{$a^0$}\underbrace{b}_\text{$b^1$}}_\text{$a^0b^1$}\underbrace{\underbrace{ }_\text{$a^0$}\underbrace{b}_\text{$b^1$}}_\text{$a^0b^1$}\underbrace{\underbrace{a}_\text{$a^1$}\underbrace{b }_\text{$b^1$}}_\text{$a^1b^1$}\underbrace{\underbrace{ }_\text{$a^0$}\underbrace{b }_\text{$b^1$}}_\text{$a^0b^1$}\underbrace{\underbrace{a}_\text{$a^1$}\underbrace{  }_\text{$b^0$}}_\text{$a^1b^0$}}_\text{$(a^*b^*)^7$}$
\end{itemize}

Le premier déroulement (on parlera de \textbf{dérivation}) semble bien sûr plus "naturel" et optimal que les deux autres. Ils sont pourtant tout aussi valides, et il sera utile en pratique d'éviter des expressions fortement ambiguës comme celle-ci.


\subsection{$+$, la disjonction}

Le symbole $+$ permet quant à lui de signifier qu'on a le choix entre plusieurs sous-regex.

\begin{example}
La regex $aa+bb$ dénote l'ensemble $\{aa,bb\}$
\end{example}

Si on a par exemple $\Sigma = \{a,b,c\}$, alors $(a+b+c)$ correspond à "n'importe quelle lettre de l'alphabet". On écrira cette expression plus simplement $\Sigma$.

\begin{example}
L'expression $\Sigma \Sigma \Sigma$, ou $\Sigma^3$, correspond à n'importe quel mot de longueur 3.
\end{example}

La disjonction peut bien sûr se combiner avec $*$ :

\begin{example}
\label{ex3}
La regex $(aa)^*+(bb)^*$ dénote l'ensemble des mots composés uniquement de a et de longueur paire, ou uniquement de b et de longueur également paire, par exemple $\epsilon$, $aa$, $bb$, $aaaa$ ou $bbbbbb$. La dérivation du dernier mot serait alors de la forme 

\[
\underbrace{\underbrace{\underbrace{bb~bb~bb~bb}_\text{$(bb)^4$}}_\text{$(bb)^*$}}_\text{$(aa)^*+(bb)^*$}
\]
\end{example}

\begin{example}
\label{ex4}
La regex $(aa+bb)^*$ dénote l'ensemble des mots de la forme "une série  (éventuellement vide) de $aa$ et $bb$", par exemple $aabb$, $aaaaaaaaaaaa$ ou $bbaabbbbbbaa$. La dérivation du dernier mot ressemblerait alors à 

\[
\underbrace{\underbrace{bb}_\text{$aa+bb$}~~\underbrace{aa}_\text{$aa+bb$}~~\underbrace{bb}_\text{$aa+bb$}~~\underbrace{bb}_\text{$aa+bb$}~~\underbrace{bb}_\text{$aa+bb$}~~\underbrace{aa}_\text{$aa+bb$}}_\text{$(aa+bb)^6$}
\]
\end{example}

\begin{exercice}
Donner un mot acceptant deux dérivations avec la regex de l'exemple \ref{ex3} (justifier en donnant les dérivations). Existe-t-il un autre mot admettant plusieurs dérivations ?
\end{exercice}

\begin{exercice}
Existe-t-il un mot acceptant plusieurs dérivations pour la regex de l'exemple \ref{ex4} ?
\end{exercice}

\begin{exercice}
Donner un mot accepté par la regex de l'exemple \ref{ex4} mais pas celle de l'exemple \ref{ex3}. Est-il possible de trouver un mot qui, à l'inverse, est accepté par la deuxième mais pas la première ?
\end{exercice}


\begin{exercice} (*)
Exprimer, en langue naturelle et de façon concise, le langage dénoté par la regex de l'exemple \ref{ex2}. Traduire ensuite ce langage en une regex non-ambiguë, c'est-à-dire où il n'y aura qu'une dérivation pour chaque mot.
\end{exercice}


Notez que dans l'exemple \ref{ex3}, on choisit d'abord de composer le mot de $a$ ou de $b$, puis la longueur. A l'inverse, on choisit dans la regex de l'exemple \ref{ex4} la longueur, puis $a$ ou $b$ pour chaque "morceau", et ce, individuellement. Pour mieux comprendre cette différence, il faut s'intéresser formellement à la mécanique des regex, qui se décompose bien évidemment entre syntaxe et sémantique.

\section{Syntaxe}

%Avant de parler de syntaxe, il faut d'abord s'arrêter sur un lexique, c'est-à-dire l'ensemble des symboles qui entreront en jeu. Comme on l'a dit précédemment, les expressions régulières se composent des lettres d'un alphabet donné, de $+$ et de $*$. On utilisera de plus les méta-symboles $($ et $)$ pour en coder la structure linéairement.

Les expressions rationnelles ont, comme à peu près tout langage, une structure. Elles sont définies à l'aide de 5 règles, dont 3 récursives, qui correspondent au lexique décrit précédemment : 

\begin{figure}[h]
    \centering
\begin{tabular}{cccl}
$e$ & $\textcolor{black}{::=}$ & & $\epsilon$\\
& & $|$&  $a \in \Sigma$\\
& & $|$&  $e_1.e_2$\\
& & $|$&  $e_1+e_2$\\
& & $|$&  $e_1^*$\\
\end{tabular}
\caption{Syntaxe des expression régulières}
    \label{resynfig}
\end{figure}

La figure \ref{resynfig} se lit "une expression rationnelle $e$ est 

\begin{itemize}
    \item[\textbf{soit}] le symbole $\epsilon$
    \item[\textbf{soit}] une lettre appartenant à l'alphabet $\Sigma$
    \item[\textbf{soit}] une expression rationnelle $e_1$ (définie à l'aide des mêmes règles), suivie de $.$, puis d'une expression rationnelle $e_2$
    \item[\textbf{soit}] une expression rationnelle $e_1$, suivie de $+$, puis d'une expression rationnelle $e_2$
    \item[\textbf{soit}] une expression rationnelle $e_1$ auréolée d'un $*$
\end{itemize}

et rien d'autre".

Ces règles de dérivation nous permettent de \textit{parser} des expressions rationnelles. En notant $t()$ la fonction qui prend une regex et renvoie son arbre syntaxique, on peut la définir à l'aide des 5 règles de la figure \ref{resynfig} :

\begin{itemize}
\item $t(\epsilon)$ renvoie une feuille annotée par $\epsilon$.
\item $t(a)$ renvoie une feuille annotée par $a$.
\item $t(e_1.e_2)$ renvoie un noeud $.$ dont les descendants sont les arbres de $e_1$ et $e_2$, comme sur la figure \ref{parseconcat}
\item $t(e_1+e_2)$ renvoie un noeud $+$ dont les descendants sont les arbres de $e_1$ et $e_2$, comme sur la figure \ref{parsedisj}
\item $t(e_1^*)$ renvoie un noeud $*$ avec un seul descendant, l'arbre de $e_1$, comme sur la figure \ref{parseit}
\end{itemize}

\begin{figure*}[t!]
    \centering
    \begin{subfigure}[b]{0.3\textwidth}
        \centering
 \raisebox{-2.2\height}{$t(e_1.e_2) =$} \raisebox{-0.5\height}{\Tree[.{.} $t(e_1)$ $t(e_2)$ ]}
    \caption{Concaténation}
    \label{parseconcat}
    \end{subfigure}%
    ~~
    \begin{subfigure}[b]{0.3\textwidth}
 \raisebox{-2.2\height}{$t(e_1+e_2) =$} \raisebox{-0.5\height}{\Tree[.{+} $t(e_1)$ $t(e_2)$ ]}
    \caption{Disjonction}
    \label{parsedisj}
    \end{subfigure}
   ~~
    \begin{subfigure}[b]{0.3\textwidth}
 \raisebox{-2.2\height}{$t(e_1^*) =$} \raisebox{-0.5\height}{\Tree[.{*} $t(e_1)$ ]}
    \caption{Etoile}
    \label{parseit}
    \end{subfigure}
    \caption{Analyse syntaxique récursive de \textit{regex}}
\end{figure*}


Les règles décrites ci-avant ne disent pas comment parser une expression comme $a.b+c$. En effet, rien ne dit si elle doit être lue comme $(ab)+c$ ou $a.(b+c)$. Pour éviter d'avoir à mettre des parenthèses absolument partout, on va donc devoir définir les priorités entre les différentes opérations :

\begin{figure}[h]
    \centering
    $+ < . < *$
    \caption{Priorités pour les opérateurs d'expressions rationnelles}
    \label{prio}
\end{figure}

Concrètement, $+ < .$ veut dire qu'une expression comme $a.b+c$ doit être interprétée comme $(a.b)+c$\footnote{De la même façon que $a \times b + c$ se comprend comme $(a \times b) + c$}. De même, $a.b^*$ se lit $a.(b^*)$, et $a+b^*$ comme $a+(b^*)$.

Maintenant qu'on a les règles de dérivation et les priorités associées, on peut commencer à jouer avec quelques exemples.

\begin{example}
\label{parsefig}
L'expression rationnelle $(aa)^*+(bb)^*$ peut être parsée comme 
\hspace{-5cm}
\begin{figure}[h!]
    \centering
    \hspace*{-2cm}\begin{tabularx}{\textwidth}{ccccccccc}
         $t((aa)^*+(bb)^*)$& $ = $ & \noindent\parbox[c]{3cm}{\Tree[.{$+$} $t((aa)^*)$ $t((bb)^*)$ ]} & $ = $ & \noindent\parbox[c]{2cm}{\Tree[.{$+$} [.{$*$} $t(aa)$ ] [.{$*$} $t(bb)$ ] ]} & $=$ & \noindent\parbox[c]{3.35cm}{\Tree[.{$+$} [.{$*$} [.{$.$} $t(a)$ $t(a)$ ] ] [.{$*$} [.{$.$} $t(b)$ $t(b)$ ] ] ]} & $ = $ & \noindent\parbox[c]{3cm}{\Tree[.{$+$} [.{$*$} [.{$.$} $a$ $a$ ] ] [.{$*$} [.{$.$} $b$ $b$ ] ] ]}
    \end{tabularx}
    \caption{Analyse syntaxique de $(aa)^*+(bb)^*$}
    \label{}
\end{figure}
\end{example}

On remarquera bien sûr l'absence habile dans l'exemple \ref{parsefig} de formes encore problématiques : $a+b+c$ et $a.b.c$. En effet, rien ne nous dit pour l'instant auquel des arbres de la figure \ref{ambfig} la première regex correspond.


\begin{figure*}[h!]
    \centering
    \begin{subfigure}[b]{0.5\textwidth}
        \centering
\Tree[.{$+$} [.{$+$} $a$ $b$ ] $c$ ]
\caption{version $(a+b)+c$}
    \end{subfigure}%
    ~ 
    \begin{subfigure}[b]{0.5\textwidth}
        \centering
\Tree[.{$+$} $a$ [.{$+$} $b$ $c$ ] ]
\caption{version $a+(b+c)$}
    \end{subfigure}
    \caption{Ambiguïté syntaxique de $a+b+c$}
        \label{ambfig}
    \end{figure*}

Comme on le verra dans la partie sémantique, les symboles $+$ et $.$ sont \textbf{associatifs}, ce qui veut dire que, pour toutes expressions $e_1$, $e_2$ et $e_3$, $(e_1 + e_2) + e_3$ et $e_1 + (e_2 + e_3)$ ont le même sens\footnote{Notez qu'en arithmétique, $+$ et $\times$ sont également associatives}, et pareil avec la concaténation. Malgré une méfiance justifiée des arbres ternaires et plus, on se permettra donc d'écrire des expressions ambiguës comme $e_1 + e_2 + e_3$ ou $e_1e_2e_3$, et de les parser comme dans la figure \ref{ambpasfig}

\begin{figure}[h!]
    \centering
    \Tree[.{$+$} $t(e_1)$ $t(e_2)$ $t(e_3)$ ]
    \caption{Ambiguïté syntaxique assumée de $a+b+c$}
    \label{ambpasfig}
\end{figure}

\begin{example}
L'expression rationnelle $(aa(a+b)^*bb)^*$ s'analyse comme 

\begin{figure}[h!]
    \centering
    \Tree[.{$*$} [.{.} $a$ $a$ [.{$*$} [.{$+$} $a$ $b$ ] ] $b$ $b$ ] ]
    \label{rien}
\end{figure}
\end{example}

%\begin{exercice}
%Donner les arbres de syntaxe des expressions $e_1 = $
%\end{exercice}

On a pour l'instant uniquement défini le lexique et la syntaxe des expressions régulières, mais les beaux arbres qu'on est désormais en mesure de constuire n'ont en soi aucun sens, et donc aucun intérêt. Il s'agit donc désormais d'en définir une sémantique.

\section{Sémantique}
\label{resem}

Avant de regarder la tuyauterie d'une fonction, il s'agit d'en définir le type. La sémantique des expressions rationnelles, notée $\llbracket e \rrbracket$, prend en argument une expression et renvoie un langage, donc un ensemble de mots. Comme pour le parsing, il suffit de définir la sémantique sur les 5 constructeurs des expressions rationnelles pour pouvoir toutes les traiter :


\subsection{Les cas de base}

Ici, pas de surprise, $\llbracket \epsilon \rrbracket = \{\epsilon\}$ et $\llbracket a \rrbracket = \{a\}$. 

\subsection{Sémantique de la concaténation}

Formellement, on a 

\[\mt{\Tree[.{.} $e_1$ $e_2$ ]} = \llbracket e_1 \rrbracket \times \llbracket e_2 \rrbracket = \bigcup_{\substack{u \in \llbracket e_1 \rrbracket \\ v \in \llbracket e_2 \rrbracket}} u.v \]

Concrètement, ça veut dire que la sémantique de la concaténation de deux regex ($\llbracket e_1.e_2 \rrbracket$) est le produit cartésien ($\times$) des sémantiques de $e_1$ ($\llbracket e_1 \rrbracket$) et $e_2$ ($\llbracket e_2 \rrbracket$), c'est-à-dire l'ensemble des combinaisons d'un mot de $\llbracket e_1 \rrbracket$ concaténé à un mot de $\llbracket e_2 \rrbracket$. La notation $\bigcup_{\substack{u \in \llbracket e_1 \rrbracket \\ v \in \llbracket e_2 \rrbracket}} u.v$ est analogue une double boucle sur les éléments de $\llbracket e_1 \rrbracket$ et $\llbracket e_2 \rrbracket$, comme dans le code python suivant :

\begin{python}
s = set()
for u in e1:
    for v in e2:
        s.add(u.v)
return s
\end{python}


\begin{example}
En appliquant les règles de la concaténation et des lettres vues précédemment, la sémantique de l'expression $ab$ est 

\[\mt{\Tree[.{.} a b ]} = \bigcup_{\substack{u \in \llbracket a \rrbracket \\ v \in \llbracket b \rrbracket}} u.v = \bigcup_{\substack{u \in \{a\} \\ v \in \{b\}}} u.v = \{a.b\} = \{ab\}\]
\end{example}

L'exemple n'est pas renversant, mais permet d'illustrer l'aspect purement systémique et récursif de la sémantique. Pour des exemples plus rigolos, on va avoir besoin d'ajouter des constructeurs à la sémantique.

\subsection{Sémantique de la disjonction}

Formellement, on a 

\[\mt{\Tree[.{+} $e_1$ $e_2$ ]} = \llbracket e_1 \rrbracket \cup \llbracket e_2 \rrbracket \]

Concrètement, ça veut dire que la sémantique de la disjonction de deux regex ($\llbracket e_1 + e_2 \rrbracket$) est l'union ($\cup$) des sémantiques de $e_1$ ($\llbracket e_1 \rrbracket$) et $e_2$ ($\llbracket e_2 \rrbracket$), c'est-à-dire l'ensemble des mots qui apparaissent dans $\llbracket e_1 \rrbracket$ ou (inclusif) $\llbracket e_2 \rrbracket$.

\paragraph*{Remarque} A partir d'ici et pour des raisons de mise en page, on ne mettra pas forcément tout sous forme d'arbres dans les exemples, et on comptera sur la capacité du lecteur ou de la lectrice à \textit{parser} automatiquement toute expression rationnelle qu'il ou elle lit. Ne vous y trompez pas cependant : l'analyse sémantique s'opère bien sur un arbre plutôt que sur une expression "plate". 

\begin{example}
En appliquant les règles de la concaténation et des lettres vues précédemment, la sémantique de l'expression $a(b+c)$ est 

\[\mt{\Tree[.{.} a [.{+} b c ] ]} = \bigcup_{\substack{u \in \llbracket a \rrbracket \\ v \in \llbracket b + c \rrbracket }} u.v = \bigcup_{\substack{u \in \{a\} \\ v \in \llbracket b \rrbracket \cup \llbracket c \rrbracket}} = \bigcup_{\substack{u \in \{a\} \\ v \in \{b,c\} }} u.v = \{a.b, a.c\} = \{ab, ac\}\]
\end{example}

\begin{example}
En appliquant les règles de la concaténation et des lettres vues précédemment, la sémantique de l'expression $(a+b)(b+a)$ est 

\[\mt{\Tree[.{.} [.{+} a b ] [.{+} b a ] ]} = \bigcup_{\substack{u \in \llbracket a + b \rrbracket \\ v \in \llbracket b + a \rrbracket }} u.v = \bigcup_{\substack{u \in \llbracket a \rrbracket \cup \llbracket b \rrbracket \\ v \in \llbracket b \rrbracket \cup \llbracket a \rrbracket}} = \bigcup_{\substack{u \in \{a,b\} \\ v \in \{b,a\} }} u.v = \{a.b, a.a, b.b, b.a\} = \{ab, aa, bb, ba\}\]
\end{example}

Il ne nous manque maintenant que le plus ésotérique des constructeurs.

\subsection{Sémantique de l'itération}

Formellement, on a 

\[\llbracket e^* \rrbracket = \bigcup_{n \in \mathbf{N}} \llbracket e \rrbracket^n = \{\llbracket e \rrbracket^0, \llbracket e \rrbracket^1, \llbracket e \rrbracket^2, \llbracket e \rrbracket^3 ...\}\]


Concrètement, on fait l'union de $\llbracket e \rrbracket^n$ pour tous les entiers $n$, $\llbracket e \rrbracket^n$ étant $n$ mots de $\llbracket e \rrbracket$ concaténés\footnote{Les puissances sur les ensembles ont le même sens que sur les nombres, avec la multiplication remplacée par la concaténation}.

\begin{example}
La sémantique de l'expression $a(aa+bb)^*a$ est
%\hspace{-5cm}
\begin{figure}[h]
\hspace{-2cm}
\begin{tabular}{cl}
$\mt{\Tree[.{.} a [.{*} [.{+} [.{.} a a ] [.{.} b b ] ] ] a ]}$ &
\begin{tabular}{ll}
$=$& $\llbracket a \rrbracket . \llbracket (aa+bb)^* \rrbracket . \llbracket a \rrbracket$ \\ & \\ $=$ & $\{a\} . \bigcup_{n \in \mathbf{N}} \llbracket aa+bb \rrbracket^n . \{a\}$ \\ & \\ $=$ & $\{a\} . \bigcup_{n \in \mathbf{N}} \{aa,bb\}^n . \{a\}$ \\ 
& \\
$=$ & $\{a.\epsilon.a, a.aa.a, a.bb.a, a.((aa).(aa)).a, a.((aa).(bb)).a, a.((bb).(aa)).a, a.((bb).(bb)).a, ...\}$ \\
&\\
$=$ & $\{aa, aaaa, abba, aaaaaa, aaabba, abbaaa, abbbba, ...\}$
\end{tabular} \end{tabular}
\end{figure}
\end{example}

\section{Mise en application}

On a abordé les expressions régulières sous un angle très théorique, mais on leur trouve bien sûr des applications concrètes. 

\subsection{Quelques astuces}

On présente d'abord, sous forme d'exercices (corrigés dans un autre document), quelques astuces classiques, susceptibles d'aider les TAListes dans leurs futures oeuvres. 

\begin{exercice}
Donner une \textit{regex} pour les mots qui commencent par $a$.
\end{exercice}

\begin{exercice}
Donner une \textit{regex} pour les mots qui finissent par $b$.
\end{exercice}

\begin{exercice}
Donner une \textit{regex} pour les mots qui commencent par $a$ finissent par $b$.
\end{exercice}


\begin{exercice}
Donner une \textit{regex} pour les mots de longueur paire.
\end{exercice}

\begin{exercice}
Donner une \textit{regex} pour les mots de longueur impaire qui contiennent au moins 4 lettres.
\end{exercice}

\begin{exercice}
Donner une \textit{regex} pour les mots de longueur impaire, qui contiennent au moins 4 lettres, comment par $a$ et finissent par $b$.
\end{exercice}


\subsection{Syntaxe en pratique}

Les expressions régulières dans Unix, Python \& cie utilisent une syntaxe différente, et surtout plus étendue que celle que l'on vient d'étudier. Celà est dû aux besoins différents que l'on a entre la théorie et la pratique. 

Dans la théorie, on veut définir nos objets de façon minimale, c'est-à-dire avec le moins de symboles et de règles possible, afin d'en simplifier l'étude. Par exemple, le peu de règles permet une définition légère de la sémantique formelle des \textit{regex}. De la même façon, toute preuve à leur propos en sera tout autant simplifiée :

\begin{theorem}
$\forall e, \exists w \in \llbracket e \rrbracket$, cad. que toute expression rationnelle dénote au moins un mot.
\end{theorem}
\begin{proof}
On procède par induction structurelle sur l'expression rationnelle $e$:

\begin{itemize}
\item Si $e = \epsilon$, alors $\llbracket e \rrbracket = \{\epsilon\}$, qui contient bien un mot ($\epsilon$ donc)
\item Si $e = a$, alors $\llbracket e \rrbracket = \{a\}$, qui contient bien un mot ($a$)
\item Si $e = e_1 + e_2$, alors $\llbracket e \rrbracket = \llbracket e_1 \rrbracket \cup \llbracket e_2 \rrbracket$. Par hypothèse d'induction, $\llbracket e_1 \rrbracket$ contient un mot $w_1$ et $\llbracket e_2 \rrbracket$ contient $w_2$. $\llbracket e \rrbracket$ contient donc non pas un, mais au moins deux mots.
\item Si $e = e_1.e_2$, alors $\llbracket e \rrbracket = \llbracket e_1 \rrbracket . \llbracket e_2 \rrbracket$. Par hypothèse d'induction, $\llbracket e_1 \rrbracket$ contient un mot $w_1$ et $\llbracket e_2 \rrbracket$ contient $w_2$. $\llbracket e \rrbracket$ contient donc un mot, $w_1w_2$.
\item Si $e = e_1^*$, alors $\llbracket e \rrbracket$ contient $\epsilon$.
\end{itemize}
\end{proof}

Dans la pratique, on préfère ne pas avoir à réinventer la roue chaque matin, la syntaxe des \textit{regex} est donc étendue en pratique. Ces extensions ne changent rien au fond, dans le sens où elles n'ajoutent pas en expressivité. En effet, les nouveaux symboles peuvent tous être codés avec ceux de la syntaxe minimale : 

\begin{itemize}
\item $e?$, ou "$e$ une ou zéro fois", peut être codée comme $(e + \epsilon)$
\item $e+$, ou "$e$ au moins une fois", peut être codée comme $ee^*$
\begin{itemize} \item [remarque] On trouve régulièrement cette notation dans la littérature académique sous la forme $e^+$. Le $e_1 + e_2$ qu'on a vu est lui, pour le coup, écrit $e_1 | e_2$ en pratique. \end{itemize}
\item $e\{n\}$, ou "$e$ exactement $n$ fois" peut être simplement traduite en $\underbrace{eee...eee}_\text{$n$ fois}$ 
\item $e\{n,\}$, ou "$e$ au moins $n$ fois" peut être simplement traduite en $\underbrace{eee...eee}_\text{$n$ fois}e^*$ 
\item $e\{n,m\}$, ou "$e$ entre $n$ et $m$ fois" peut se traduire $\underbrace{eee...eee}_\text{$n$ fois}\underbrace{(e+\epsilon)...(e+\epsilon)}_\text{$m-n$ fois}$
\end{itemize}

Les traductions proposée ici ne correspondent pas forcément à ce qui se passe concrètement dans les bibliothèques de \textit{regex} des différents langages de programmation (la dernière en particulier semble très ambigüe, et donc inefficace). L'objectif est seulement de montrer que les ajouts à la syntaxe n'ent modifient pas l'expressivité, et qu'il s'agit seulement de \textbf{sucre syntaxique}.
