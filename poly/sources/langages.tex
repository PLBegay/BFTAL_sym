
\chapter{Langages}
\label{langages}

On définit d'abord la notion de mot, nécessaire à celle de langage. On verra ensuite comment décrire des langages à l'aide de notations ensemblistes, révisant ces dernières par la même occasion.

\section{Mots}

\begin{definition}{Mot}{}Un \textbf{mot} est une suite de lettres tirées d'un alphabet donné. L'ensemble des mots sur un alphabet $\Sigma$ est noté $\Sigma^*$.
\end{definition}

\begin{example}
Etant donné l'alphabet $\Sigma = \{a,b,c\}$, on peut construire une infinité de mots parmi lesquels 

\begin{itemize}
    \item $abc$
    \item $aab$
    \item $cc$
    \item $abcabcacbacbacbacbabcabcabcabcabcabcabc$
    \item $a$
\end{itemize}

\end{example}


\paragraph{Remarque} On va s'intéresser ici à des langages et mots complètement abstraits, en général composés uniquement de $a$, $b$ et $c$.

\begin{definition}{Mot vide}{}
Une suite de lettres peut être de longueur zéro, formant alors \textbf{le mot vide}. Quel que soit l'alphabet, ce dernier sera noté $\epsilon$.
\end{definition}

\begin{definition}{Concaténation}{}
\label{concat}
L'opération de \textbf{concaténation}, notée $.$, consiste tout simplement à "coller" deux mots.
\end{definition}

\begin{example}{Quelques concaténations :}

\begin{itemize}
    \item $ab.c = abc$
    \item $ab.ba = abba$
\end{itemize}
\end{example}

 De plus, pour tout mot $w$, 

\[
    w.\epsilon = \epsilon.w = w
\]

\paragraph{Remarque} Les algébristes enthousiastes remarqueront que $(\Sigma^*,.,\epsilon)$ forme un monoïde libre de base $\Sigma$


\begin{definition}{Longueur d'un mot}{}
Etant donné un mot $w$, on note sa \textbf{longueur} $|w|$.
\end{definition}

\begin{example}
Tout naturellement,
\begin{itemize}
    \item $|abc| = 3$
    \item $|abba| = 4$
    \item $|c| = 1$
    \item $|\epsilon| = 0$
\end{itemize}
\end{example}

\begin{definition}{Principe d'induction sur un mot}{}
Etant donnée une propriété $P$ sur les mots. Si on a\\

\begin{enumerate}
\item $P(\epsilon)$ (cad. que $P$ est vraie pour le mot vide)
\item $\forall w, \forall c \in \Sigma, (P(w) \rightarrow P(c.w))$ (cad. que si $P$ est vraie pour un mot, alors elle reste vraie si on rajoute n'importe quelle lettre à gauche du mot)\\
\end{enumerate}

Alors la propriété $P$ est vraie pour tout mot $w$.
\end{definition}

\paragraph*{Remarque} Est également valide le principe d'induction où, dans le cas récursif, la lettre est rajoutée à droite du mot plutôt qu'à sa gauche.


\begin{definition}{Nombre d'occurrences d'une lettre}{}
Etant donné un mot $w$ et une lettre $a$, on note $\mathbf{|w|_a}$ le nombre de $a$ dans $w$.
\end{definition}

\begin{example}
On a 
\begin{itemize}
    \item $|abc|_a = 1$
    \item $|abba|_b = 2$
    \item $|c|_a = 0$
    \item $|\epsilon|_a = 0$
\end{itemize}
\end{example}

\begin{definition}{Préfixe}{}
Un mot $p$ est un \textbf{préfixe} du mot $w$ ssi $\exists v, w = p.v$, cad. ssi $w$ \textit{commence} par $p$.
\end{definition}

\begin{definition}{Suffixe}{}
Un mot $s$ est un \textbf{suffixe} du mot $w$ ssi $\exists v, w = v.s$, cad. ssi $w$ \textit{finit} par $p$.
\end{definition}

\begin{example}
Le mot $abba$ admet comme préfixes $\epsilon$, $a$, $ab$, $abb$ et $abba$. Ses suffixes sont, quant à eux, $\epsilon$, $a$, $ba$, $bba$ et $abba$.
\end{example}

\begin{lemma}
$\forall w \in \Sigma^*, \epsilon$ et $w$ sont des préfixes de $w$
\end{lemma}
 
\begin{proof}
Pour $\epsilon$, il suffit de prendre $v = w$. A l'inverse, en prenant $v = \epsilon$, on voit que $w$ est son propre préfixe.
\end{proof}

\begin{lemma}
$\forall w \in \Sigma^*, \epsilon$ et $w$ sont des suffixes de $w$
\end{lemma}

\begin{proof}
Analogue au lemme précédent.
\end{proof}


\begin{exercice}\label{expref}
Combien de préfixes et suffixes admet un mot $w$ quelconque ?
\end{exercice}

\begin{definition}{Facteur}{}
Un mot $f$ est un \textbf{facteur} du mot $w$ ssi $\exists v_1 v_2, w = v_1.f.v_2$, cad. ssi $f$ apparaît dans $w$.
\end{definition}

\begin{example}
Les facteurs du mot $abba$ sont $\epsilon$, $a$, $b$, $ab$, $ba$, $abb$, $bba$ et $abba$. 
\end{example}

\begin{lemma}
$\forall w \in \Sigma^*$, $\epsilon$ et $w$ sont des facteurs de $w$.
\end{lemma}

\begin{proof}
Pour $\epsilon$, il suffit de prendre $v_1 = w$ et $v_2 = \epsilon$ (ou l'inverse) et la condition est trivialement vérifiée. Pour $w$, on prend $v_1 = v_2 = \epsilon$.
\end{proof}

\begin{exercice}
Donner l'ensemble des facteurs du mot $abbba$.
\end{exercice}

\begin{exercice} \label{exfact}(*)
Donner la borne la plus basse possible du nombre de facteurs d'un mot $w$. Donner un mot d'au moins 3 lettres dont le nombre de facteurs est exactement la borne donnée.
\end{exercice}

\begin{definition}{Sous-mot}{}
Un mot $s$ est un \textbf{sous-mot} du mot $w$ ssi $w = v_0s_0v_1s_1v_2...s_nv_n$ et $s = s_0s_1...s_n$, cad. ssi $w$ est "$s$ avec (potentiellement) des lettres en plus".
\end{definition}

\begin{example}\label{ex5} On \underline{souligne} les lettres originellement présentes dans le sous-mot :
\begin{itemize}
   \item $ab$ est un sous-mot de $b\underline{a}a\underline{b}$, qu'on pourrait aussi voir comme $ba\underline{ab}$
   \item $abba$ est un sous-mot de $ba\underline{a}abaa\underline{bb}a\underline{a}$.
   \item $ba$ \underline{n}'est \underline{pas} un sous-mot de $aaabbb$ (l'ordre du sous-mot doit être préservé dans le mot) 
\end{itemize}
\end{example}


\begin{lemma}
$\forall w \in \Sigma^*$, $\epsilon$ et $w$ sont des sous-mots de $w$.
\end{lemma}

\begin{proof}
Pour $\epsilon$, il suffit de prendre $n = 0$, $s_0 = \epsilon$, $v_0 = w$ et $v_1 = \epsilon$ (ou l'inverse) et la condition est trivialement vérifiée. Pour $w$, on prend $n = 0$, $s_0 = w$ et $v_0 = v_1 = \epsilon$.
\end{proof}

\begin{exercice}
Montrer que tout facteur d'un mot en est également un sous-mot. A l'inverse, montrer qu'un sous-mot n'est pas forcément un facteur. 
\end{exercice}

\begin{exercice}
Donner toutes les façons de voir $abba$ comme sous-mot de $baaabaabbaa$ (cf. exemple \ref{ex5}).
\end{exercice}

\begin{exercice}
Donner l'ensemble des sous-mots de $abba$
\end{exercice}

\begin{exercice}\label{exssmot} (*)
Donner la borne la plus basse possible du nombre de sous-mots d'un mot $w$. Donner un mot dont le nombre de sous-mots est exactement la borne donnée.
\end{exercice}

\begin{exercice} (*) Dans l'exercice \ref{expref}, on demande le nombre exact de préfixes et suffixes d'un mot, alors que dans les exercices \ref{exfact} et \ref{exssmot}, on demande une borne, pourquoi ?
\end{exercice}

\section{Langage}

\begin{definition}{Langage}{}
Un langage, c'est un ensemble de mots. 
\end{definition}

On distingue donc d'entrée les deux langages extrêmes : $\mathbf{\Sigma^*}$, l'ensemble (infini) de tous les mots formés à partir de $\Sigma$, et $\mathbf{\emptyset}$, le langage / ensemble vide, qui se caractèrise comme ne contenant aucun élément.

\paragraph{Remarque} Ne surtout pas confondre $\emptyset$ et $\{\epsilon\}$. Le premier est un ensemble vide, contenant donc \textbf{0} élément, trandis que le second contient \textbf{1} élément, le mot ide. 

La question maintenant est maintenant de savoir comment on définit et parle de langages précis et plus "intermédiaires" que les deux précédents. En tout généralité, les ensembles peuvent être définis de façon \textbf{extensionnelle} ou \textbf{intentionnelle}. 

\begin{definition}{Définition extensionnelle d'un ensemble}{}
On \textbf{définit extensionnellement un ensemble} en en donnant la liste des éléments. L'ensemble vide se note quant à lui $\emptyset$.
\end{definition}

\begin{example} On définit par exemple l'ensemble (sans intérêt) suivant :
\[
    A = \{b, aca, abba\}
\]
\end{example}

Les définitions extensionnelles ont le mérite d'être pour le moins simples, mais pas super pratiques quand il s'agit de définir des ensembles avec un nombre infini d'éléments, comme l'ensemble des mots de longueur pair. 

\begin{definition}{Définition intensionnelle d'un ensemble}{}
On \textbf{définit intensionnellement un ensemble} à l'aide d'une propriété que tous ses éléments satisfont. Étant donnés une propriété $Q(x)$ (typiquement représentée sous la forme d'une formule logique) et un ensemble $A$, on note $\{x \in A~|~Q(x)\}$ l'ensemble des éléments de $A$ qui satisfont $P$. Si l'ensemble $A$ est évident dans le contexte, on s'abstiendera de le préciser.
\end{definition}

\begin{example}
On peut définir l'ensemble des mots de longueur paire $\{w \in \Sigma^*~|~|w|~pair\}$.
\end{example}

Si les définitions intentionnelles permettent, contrairement aux extensionnelles, de dénoter des ensembles contenant une infinité de mots, elles sont avant tout un outil théorique. En effet, une propriété comme "$|w|$ paire" ne dit rien à un ordinateur en soi, et doit donc être définie formellement. Se pose alors la question d'un langage pour les propriétés.

Plusieurs logiques équipées des bonnes primitives peuvent être utilisées, mais les traductions sont rarement très agréables. Certaines propriétés nécessitent en effet de ruser contre le langage, voire sont impossibles à formaliser dans certaines logiques. Il existe heureusement un outil qui va nous aider, avec le premier problème du moins.

\chapter{Expressions régulières}
\label{regex}
Les expressions régulières permettent définir de façon finie - et relativement intuitive - "la forme" des mots d'un langage, potentiellement infini. On en présentera d'abord le lexique et l'idée générale à l'aide d'exemples, puis on en définira formellement la syntaxe et la sémantique.
