\chapter{Automates finis}
\label{automates}


%\epigraph{- "Qu'est-ce qu'un automate ?"\\- "Des ronds et des flèches m'sieur !"\\- "Rien à voir ! [...] Un automate, c'est un graphe orienté."}{[censuré]}

Les automates forment un langage de programmation un peu particulier, en ce qu'il est très visuel (chaque programme est un graphe annoté, ou plus prosaïquement des ronds et des flèches) et que tout programme a le même type : un mot en entrée, un booleen en sortie. Un automate définit donc un langage en donnant un moyen automatique de déterminer si n'importe quel mot donné en fait partie ou non\footnote{En ce sens, les automates sont des \href{https://fr.wikipedia.org/wiki/Fonction_caract\%C3\%A9ristique_(th\%C3\%A9orie_des_ensembles)}{fonctions caractéristiques}, qui sont aux ensembles ce que les videurs sont aux boîtes de nuit.}.

Les automates se divisent en de nombreuses sous-catégories, dont seulement certaines ramifications seront explorées dans ce cours. On verra d'abord le fonctionnement général des automates finis déterministes (\ref{DFA}) et non-déterministes (\ref{NDFA}), en allant à chaque fois du général au technique. On verra ensuite des algorithmes pour transformer (\ref{transauto}) des automates. On étudiera enfin les combinaisons d'automates, et donc les propriétés de clôture des langages qu'ils définissent (\ref{cloture}).

\section{Automates finis déterministes}
\label{DFA}

\subsection{Principe général}

\subsection{Formalisation}

\section{Automates finis non-déterministes}
\label{NDFA}

\subsection{Principe général}

\subsection{Formalisation}

\section{Transformation d'automates}
\label{transauto}

\subsection{Complétion}

\subsection{Déterminisation}

\subsection{Minimisation}

\section{Propriétés de clôture}
\label{cloture}
\subsection{Union}

\subsection{Intersection}

\subsection{Concaténation}

\subsection{Itération}


