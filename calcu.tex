
\chapter{Notion de calculabilité}

%Pour une version plus détaillée de cette section, voir \cite{dowek}. 
Pour une version plus détaillée, voir \href{https://www.college-de-france.fr/site/xavier-leroy/inaugural-lecture-2018-11-15-18h00.htm}{la leçon inaugurale de Xavier Leroy au Collège de France}.

\section{Différents modèles de calcul}

L'informatique a vocation à automatiser les calculs afin de les faire réaliser par une machine plutôt que par des Humains. Avant d'automatiser les calculs, il s'agit donc de définir formellement de quoi on parle.

D'un point de vue programmation, 2 principaux modèles co-existent : les \textbf{machines de Turing} et le \textbf{$\lambda$-calcul}. Ces langages ne sont pas utilisables en pratique (\href{http://www.ens-lyon.fr/actualite/lecole/la-machine-de-turing-en-legos}{encore que}), mais posent les fondamentaux de ce qu'est un langage de programmation.

\paragraph{Machines de Turing} Les machines de Turing disposent d'une notion d'état, et d'une mémoire infinie modifiable. La notion d'état sera discutée en longueur dans la section \ref{automates}, mais correspond en gros à une mémoire spéciale, propre à chaque programme, qui peut être modifiée et consultée facilement, notamment pour s'orienter dans le \textit{flow} du programme\footnote{Pensez à une variable booleene "premierefois" que vous avez sans doute déjà utilisée dans un \texttt{if} pour accéder ou non à un cas particulier}. Ces traits rapprochement fortement les machines de Turing de la programmation impérative (C, langage machine, le coeur de Python etc). Les machines de Turing sont dues à \textbf{Alan Turing}.

\paragraph{$\lambda$-calcul} A la différence des machines de Turing, qui ont une approche quasiment mécanique (pour ne pas dire "bidouille") de l'exécution d'un programme, le $\lambda$-calcul est profondément mathématique. Tout n'y est que fonction, au point que ces dernières sont des objets comme les autres, notamment passables en arguments. On citera l'exemple classique d'une fonction qui reçoit une fonction de tri et une liste, et renvoie la liste triée selon la fonction fournie. Le $\lambda$-calcul est la base de la programmation fonctionnelle. Il a été crée par \textbf{Alonzo Church}.

\paragraph{Remarque} Les types en programmation impérative n'ont souvent qu'une valeur de garde-fou contre des opérations totalement absurdes, alors qu'ils ont une fonction beaucoup plus structurante (certaine.s diraient "contraignante") en programmation fonctionnelle. L'utilisation de fonctions comme arguments oblige par exemple à repenser les types et aller plus loin que les classiques \verb!bool!, \verb!int! et cie. On renverra encore une fois à la présentation de Xavier Leroy citée en introduction pour une meilleur vision d'ensemble.

Ces deux modèles ne forment pas l'alpha et l'omega de la calculabilité, qui contient de nombreux modèles plus ou moins exotiques, comme les fonctions $\mu$-récursives ou les automates cellulaires (voir à ce sujet \href{https://www.youtube.com/watch?v=S-W0NX97DB0}{la vidéo de la chaîne ScienceEtonnante}).

\paragraph{Thèse de Church (ou thèse de Church-Turing)} Church et Turing ont montré, dans les années 30, que les machines de Turing et le $\lambda$-calcul sont équivalents, dans le sens où toute fonction exprimable dans un modèle le sera dans l'autre. On dit que les modèles ont la même \textbf{expressivité}. Attention cependant, certaines fonctions très simples en $\lambda$-calcul seront un enfer à coder en machine de Turing, et inversement\footnote{Penser à la différence entre compétence et performance.}. Mais il reste remarquable que deux modèles fonctionnant de façons si orthogonales aient, au fond, la même puissance. 

Ce résultat est moins surprenant avec le recul, puisqu'on sait maintenant que tous la plupart des modèles de calcul non-triviaux sont équivalents, et forment la classe des modèles \textbf{Turing-complets}. Il est en effet possible de simuler, ou coder, les machines de Turing ou le $\lambda$-calcul dans les fonctions $\mu$-récursives ou les automates cellulaires, et inversement. Ce résultat s'étend à une armée de modèles, auxquels ils faut ajouter aujourd'hui des milliers de langages de programmation : Python, C, OCaml, Java, et même Makefile, Bash ou $\LaTeX$ sont bel et bien aussi expressifs les uns que les autres.

La thèse de Church peut même s'étendre à l'épistémologie ou à la philosophie, puisqu'elle peut sembler suggérer l'existence d'une notion "naturelle" et indépassable de calcul. On conseillera la lecture de \cite{dowek} pour une introduction à ces problématiques.


Le $\lambda$-calcul est utilisé en NLP, mais sera étudié dans un autre cours. Les machines de Turing quant à elles n'ont, en soi, pas d'intérêt pour la linguistique, mais on peut les affaiblir pour les rendre paradoxalement plus pertinentes.

\section{Décidabilité}

Au $XVII^{ème}$ siècle, Leibniz rêve d'une procédure permettant de déterminer automatiquement, \textit{via} un calcul, si une formule mathématique est vraie ou non. Leibniz se rendit compte que les bases formelles n'étaient alors pas disponibles, notamment la formalisation du calcul. Le problème réapparaît dans un cadre plus faorable, en 1928, lorsque Hilbert repose la question dans le cadre de son fameux programme de refondation des mathématiques. Le problème de la décision prend alors le doux nom d'\textbf{Entscheidungsproblem}. 

La réponse ne se fera pas (trop) attendre, et c'est un double non. Church et Turing publient en 1936, mais indépendament, des preuves qu'une telle procèdure, ou plutôt un tel programme, est impossible à écrire en machine de Turing et $\lambda$-calcul, et donc pour tout modèle de calcul équivalent. Le problème de la décision est \textbf{indécidable}, et il est loin d'être le seul.

\begin{theorem}{\textbf(Indécidabilité du problème de l'arrêt)}} Savoir si un programme termine est un problème indécidable.
\end{theorem}

\begin{proof}
On va procéder par l'absurde. La décidabilité du problème de l'arrêt signifierait qu'il existe un programme, appelé $A$, qui prend en argument un programme $P$ et un élément $x$, et renvoie \verb!true! si et seulement si $A(x)$ termine.

A partir de $A$, on peut constuire un autre programme appelé $B$, qui prend en argument un programme $P$, et termine si et seulement si $P(P)$ ne termine pas :\\
\verb!def B P := if (A P P) then (while true skip) else skip!.

Maintenant, appliquons $B$ à lui même. On utilisant la définition de B, on obtient\\ \verb!B(B) := if (A B B) then (while true skip) else skip!, ce qui veut dire que $B(B)$ termine si et seulement si $B(B)$ ne termine pas. On obtient donc un paradoxe, signifiant que notre seule hypothèse, l'existence de $A$, est fausse.
\end{proof}

\paragraph{Remarque} La preuve contient une bizarrerie, à savoir l'application d'un programme à lui-même ($B(B)$). Une telle chose est proscrite par l'utilisation de types, qui rendent la preuve donnée caduque. Il est cependant toujours possible de prouver l'indécidabilité de l'arrêt pour des programmes typés, de façon plus tordue cependant.

Au-delà de ces deux problèmes particulièrement connus, il existe une véritable armée de problèmes indécidables, comme le montre spécifié par le théorème de Rice :


\begin{theorem}{\textbf(Théorème de Rice)}} On appelle propriété sémantique non-triviale une propriété sur le comportement d'un programme telle qu'il existe au moins un exemple la respectant et un ne la respectant pas. Toute propriété sémantique non-triviale est indécidable.
\end{theorem}

\begin{proof}
On procède encore une fois par l'absurde, en supposant qu'il existe une propriété sémantique non-triviale $i$ décidable. Puisque $i$ est non-triviale, on sait qu'il existe $P_{i+}$ (resp. $P_{i-}$) un programme qui satisfait (resp. ne satisfait pas) la propriété $i$. On va montrer qu'il est alors possible de résoudre le problème de l'arrêt.

Soit un programme $P$ dont on veut vérifier qu'il termine sur l'argument $x$. On vérifie d'abord si $P$ satisfait la propriété $i$. Supposons, sans perte de généralité (il suffit sinon d'inverser les $+$ et $-$), que ce n'est pas le cas. On écrit alors un programme qui fait tourner $P(x)$, puis $P_{i+}$. On vérifie si le tout satisfait $i$. Si c'est le cas, on sait que $P(x)$ a fini. A l'inverse, si ce n'est pas le cas, $P_{i+}$ n'a pas été atteint, ce qui veut dire que $P(x)$ n'a pas fini.

L'existence supposée de la décidabilité propriété sémantique non-triviale permet de 
résoudre le problème de l'arrêt, pourtant indécidable. Il n'existe donc pas de telle propriété.\end{proof}

Intuitivement, tous ces problèmes sont indécidables sur tout modèle Turing-complet car ces derniers sont trop puissants, ou expressifs\footnote{Comme le disent tous les oncles Ben du monde, un grand pouvoir (expressif) implique une grande indécidabilité}. Commence alors un jeu consistant à affaiblir les modèles pour qu'on puisse décider des propriétés à leur sujet, sans pour autant qu'ils en deviennent trivial. Ce cours va s'intéresser à la famille des automates, particulièrement adaptée à l'analyse linguistique.%De cette recherche découle notamment un modèle qui 1) est plus faible 2) concerne les langages, et donc les linguistes : les automates finis. 

Puisqu'on veut les analyser, on va d'abord définir la notion de langages - et donc de mots dans la section \ref{langages}. On étudiera dans la section \ref{regex} un outil pour les décrire et manipuler, les expressions régulières. On pourra ensuite s'intéresser aux automates les plus simples dans la section \ref{automates}, ainsi qu'à la notion connexe de grammaires formelles en \ref{grammaires}. Enfin, on étudiera en \ref{hierarchie} la façon dont ces différents outils s'imbriquent, et on \textit{teasera} la suite du cours, en évoquant des extensions des automates.
